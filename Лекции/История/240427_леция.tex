\documentclass{article}

\usepackage[T2A]{fontenc}
\usepackage[utf8]{inputenc}
\usepackage[english,russian]{babel}

\begin{document}
\section{Советский союз в послевоенный период: экономика, внешняя политика}
\subsection{Политический режим}
\par С формальной точки зрения политический режим практические не изменился с довоенного периода.

С 1945 года распускается ГКО, высшим органом власти считается Верховный совет СССР. Он оформлял в виде указов решения, принимаемые высшим партийным руководством.

Практически всегда всё одобрялось единогласно.

В период между сессиями президиум Верховного совета СССР брал на себя эти функции. В состав президиума входили члены ЦК ВКП(б) и Политбюро ВКП(б).

Вместо наркомов вводились министерства. 

Наиболее оверенные лица Сталина --- Георгий Максимилианович Маленков, Лаврентий Павлович Берия, Никита Сергеевич Хрущёв, Вячеслав Михайлович Молотов, Николай Александрович Булгарин.

Особое значение имели МВД и МГБ, возглавляемые Берия.

Ленинградское дело, Мингрельское дело, дело еврейского антифашистского комитета --- крупные политические дела этого периода.

В годы войны в рамках борьбы против фашизма был создан еврейский антифашистский комитет. Эта организация стала вызывать подозрения после войны, поэтому его лидеры были арестованы, а организация --- закрыта.

Мингрельское дело касалось западной Грузии.

Ленинградское дело --- крупнейшее. Лидеры партийной организации Ленинградской области обвинялись в подрывной деятельности. Наиболее видные деятели были расстреляны, прочие арестованы. Более двух тысяч коммунистов подверглись чистке.

Как считают историки, эти дела были инсценировкой.

Наиболее активные очаги реального сопротивления --- западная кукраина, где действовали бандеровцы. В годы Великой Отечественной эта организация сотрудничала с гитлеровскими властями и поощрялась Германией.

Ещё один регион --- Литва, где лесные братья вели партизанскую борьбу. В годы Великой Отечественной они также сотрудничали с нацистами.

В ходе действий нацистов на территории северо-восточной Европы было захвачено немало участников белого движения и антисоветских организаций. Многие из них сотрудничали с нацистами, поэтому подвергались особо жестокому преследованию со стороны советских властей.

Над пособниками оккупантов проводились процессы. Суд состоялся и над руководителями российской освободительной армии во главе с генералом Власовым.

Развитие науки и культуры было направлено исключительно на удовлетворение потребностей социалистического строительства. Расходы на образование и науку были существенно увеличены из-за особо острого дефицита специалистов.

Создаются крупные научно-исследовательские институты: институт точной механики и вычислительной техники, институт атомной энергии, институт прикладной физики и многие другие. Многие из них были совершенно новые и передовые.

В начале 1950-х годов советская наука и техника быстро прогрессирует, но это были отрасли, так или иначе связанные с военными нуждами.

В 1949 году Советский Союз впервые испытал ядерное оружие. Велась разработка химического и бактериологического оружия.

Тем временем генетиков и кибернетиков обвинили в противоречии идеям материализма и провозгласили лженаукой. В этих отраслях впоследствии Советский Союз серьёзно отстал.

Развернулась кампания по борьбе с космополитизмом. Цель борьбы --- очернить всё несоветское и несоциалистическое. Удар на себя испытала та часть советской интеллигенции, которая оглядывалась на тенденции запада. Такие писатели, учёные, литераторы и прочие оказывались либо под вниманием репрессивных органов, либо отстранёнными от своей работы. Это приводило к падению художественно-идейного уровня.

\subsection{Внешняя политика}
Возросла роль Советского Союза в международных отношениях. В Европе происходят территориальные изменения. Сократилась территория Германии, часть её пространств отошла Польши. Пруссию разделили: юг также ушёл Польше, а север с городом Кёнигсберг вошёл в состав РСФСР. СССР вернул часть пространств, присоединённых в сентябре 1939 года.

К Советскому Союзу по договору с Чехословакией отошла Закарпатская область, населённая украинцами и русинами.

По условиям Ялтинской конференции Советский Союз получил южный Сахалин и Курильские острова. 

Администрация Гарри Трумэна приняла план Маршалла, по которому Америка помогала восстановлению западноевропейской экономики технологиями, кредитами, оборудованием. Многие западноевропейские страны воспользовались этим и подверглись влиянию США.

В восточной Европе формировались государственные системы, аналогичные советскому. По такому пути пошли Польша, Чехословакия, Венгрия, Румыния, Болгария, Албания и Югославия (Югославская Советская Федеративная Социалистическая Республика, в которую входило несколько республик).

Во главе Югославии стоял Иосип Броз Тито, который начал конфронтацию со Сталиным в 1948 году.

Германия была разделена на зоны оккупации. В результате Германия распалась на просоветскую Германскую демократическую республику (ГДР), проамериканскую федеративную республику Германию и западный Берлин, территориально отделённый от восточного, являвшегося столицей ГДР.

Послевоенная Европа разделилась на две противостоящие друг другу группы государств. В 1949 году возникает Североатлантический альянс, на востоке Европы возникает другой союз, который оформится в военно-политический блок в 1955 году. А пока возник совет экономической взаимопомощи --- ответ СССР на план Маршалла. Через него Советский Союз оказывал экономическую поддержку тем странам, где к власти пришли просоветские режимы.

По масштабам и эффективности с американским планом она не могла сравниться из-за разрухи в самом СССР. План Маршалла предлагался и восточноевропейским странам, включая СССР, но Сталин не мог на это пойти, потому что он открыто требовал от Советского Союза политических реформ в сторону сближения с западными режимами.

Уистон Черчилль в городе Фултон в 1946 году произнёс речь в присутствии президента Трумэна. Он обвинял СССР в намерении уничтожить свободы и права человека. Фултонская речь положила начало советско-западной конфронтации. В идеологии двух сторон стали употребляться словосочетания "холодная война" (страны не воюют напрямую, но участвуют в прокси-войнах и всячески пытаются подорвать мощь друг друга), "железный занавес".

В 1946-1949 гг. крупным локальным конфликтом была война в Китае. К 1949 году коммунистическая партия Китая одержала верх, Гоминьдан проиграл. Китай повысил влияние Советского Союза в регионе, стал на какое-то время дружественной СССР стороной. Разбитые силы Гоминьдана бежали на Тайвань и заручились поддержкой США.

Апогей противостояния было достигнут в Корейской войне 1955-1956 гг. Советский Союз поддерживал просоветские силы, а США --- проамериканские. Оказывалась не только экономическая, но и военная поддержка: так, советские и американские пилоты непосредственно участвовали в сражениях. Советское и американское правительство опасались последствий. Здравый смысл возобладал. Корейская война была завершена перемирием и установлением границы между Северной и Южной Кореями.

\section{Личность и деятельность Сталина в исторической науке}
Сталин родился то ли 6, то ли 9 декабря 1878 года. По официальной версии это 9 декабря. Сталин уверял, что это неточность и реальный день рождения его --- 6 декабря.

Он родился в городе Гори в семье сапожника. Иосиф и его мама неоднократно были жертвами домашнего насилия. Мама отдала Иосифа в духовное училище, позже он поступил в Тифлисскую духовную семинарию. Он хорошо учился и даже писал стихи, но он не хотел быть священником.

Во время обучения в семинарии Сталин познакомился с трудами Маркса. Иосиф Джугашвили читал теоретиков, стал собирать кружки рабочих и объяснять им идеи Маркса. В 1899 году Сталин был исключён из семинарии из-за неявки на экзамен.

После этого он занялся революционной работой: изданием нелегальных газет, организацией акций протеста. Был образован Тифлисский комитет РСДРП. Иосиф пользовался подпольной кличкой Коба.

Джугашвили присоединился к большевикам, что было нехарактерно для грузин. Джугашвили лично познакомился с Лениным. Владимир Ильич уже тогда его высоко оценил.

Коба также занимался экспроприацией, за что подвергался арестам, ссылке, но при этом делал партийную карьеру. В 1912 году Сталина приняли в ЦК РСДРП(б). Он окончательно выбрал псевдоним Сталин.

В 1913 году Сталин был сослан в Сибирь в Талуханский край. Талуханск был самым маленьким городом Российской империи. Здесь он находился до самой Февральской революции.

Сталин приезжает в Петроград. До приезда Ленина он был одним из лидеров большевиков. Был одним из активных организаторов Октябрьской революции.

Во время Гражданской войны Сталин входил в Реввоенсовет республики, временами руководил целыми фронтами Гражданской войны.

Второй раз женился на Анастасии Алилуевой. Оба сына были участниками Великой Отечественной.

Яков старший был артиллеристом и попал в плен в июле 1941 года. В 1943 году при попытке к бегству был застрелен охраной.

Второй сын был военным лётчиком, занимал командные посты в этой сфере. Из-за развенчания культа личности судьба его была сложной, но умер он своей смертью. По официальной версии --- от алкоголя.

В 1932 году жена Сталина покончила с собой, возможно, из-за нездоровой обстановки в семье.

Сталин стал генеральным секретарём партии --- занял техническую должность, которую не хотели брать другие вожди. Эта работа помогла расставить ему на всех постах в стране своих личных сторонников. В начале 1920-х Сталин заявляет о себе как о возможном преемнике Ленина.

Сталин обнаружил, что, чтобы быть лидером, он должен быть не только практиком, но и идеологом, поэтому он начал писать статьи.

Современники отмечали следующие характеристики: эгоизм, пытливость, настойчивость, знание человеческих слабостей. Для него были характерны однозначность суждений, упрямство, воля к достижению целей, мстительность. В результате внутрипартийной борьбы он превратился в главного лидера партии. Он умело использовал своих противников друг против друга.

Вряд ли была сфера, в которой Сталин разбирался досконально. Троцкий называл Сталина выдающейся посредственностью. Его работы пользовались спросом только во время его правления только внутри страны.

По поводу его военных способностей также есть различные мнения. По мнению маршалов Победы, Сталин не отличался большими способностями командующего в начале войны, но зато он быстро учился и набирался опыта.

Сталин приложил руку к формированию культа. В 1938 году вышла его книга "Краткий курс истории ВКП(б)", где отражался его взгляд на историю России и Советского Союза. Чуть позже вышла и краткая биография Сталина, к которой он также приложил руку. В этих книгах Сталин освещался исключительно с положительных сторон как редко одарённая личность.

Эти книги становились набором непреложных истин. Сталину приписывались все заслуги на пути строительства социализма.

Отношение к личности Сталина после Победы достигло апогея. Стали издаваться его работы и сочинения. Тиражи Сталина вдвое превзошли тиражи Ленина.

Роль Сталина в формировании тоталитарной системы велика, но бессмысленно возлагать её только на Сталина. В формировании советской политической системы принимали участие все лидеры партии. Все, даже беспартийные советские граждане, искренне поддерживали Сталина. 

\section{Политическое развитие страны в период оттепели. 1953--1964 гг. Н. С. Хрущёв. Отношения СССР с другими странами}
Смерть Сталина наступила 5 марта 1953 года и поставила вопрос о лидерстве в партии и государстве.

Состоялось необычное совместное заседание всех трёх органов. Пристутсвовали ЦК партии, совет министров и президиум Верховного совета. Были потеряны генсек партии и председатель совета министров.

На этом заседании председателем совета министров стал Георгий Максимилианович Маленков, Лаврентий Павлович Берия сохраняет прежние должности и становится заместителем Маленкова. Министерство обороны возглавил Николай Александрович Булгарин. Булгарин и Каганович также стали заместителями Маленкова. Ворошилов возглавил президиум Верховного совета, Никита Сергеевич Хрущёв возглавил генеральный секритариат ЦК партии.

Все боялись Лаврентия Павловича Берия, потому что он возглавлял репрессивный аппарат и обладал чувствительной информацией, которую он мог бы пустить вход против своих соперников. Поэтому проищошёл заговор, на заседании ЦК партии в адрес Берия прозвучали различные обвинения. Жуков с группой вооружённых людей арестовал Берия на этом же заседании, позже были арестованы его ближайшие сторонники. Берия был приговорён к расстрелу, который состоялся в декабре 1953 года.

Устранение Берия укрепило положение Хрущёва. Чтобы избежать появления нового "Берия" было решено провести реформу. Вместо МГБ был создан КГБ, подчиняющийся совету министров СССР.

К концу 1954 года резко обострилась борьба между ближайшими соратниками Сталина. Столкнулись старые сталинисты и Хрущёв со своими сторонниками в партии и в армии. Хрущёв добился отставки Маленкова в 1955 году, его заменил Булгарин --- человек Хрущёва, --- а пост военного министра занял Жуков.

Влияние Хрущёва в новом составе руководства страны становилось преобладающим. Заседания президиума ЦК становятся регулярными, там обсуждаются не только партийные, но и экономические проблемы. Берётся курс на реабилитацию жертв сталинизма. Именно за уменьшение контроля и строгости со стороны государства эта эпоха была прозвана оттепелью.

В марте 1953 года предложение прекратить пропаганду культа личности Сталина внёс Маленков, буквально через несколько дней после смерти Сталина. Он же снизил налоги с крестьянства, аннулировал многие долги колхозов. Развивалась торговая сеть страны. Эти меры были направлены на повышение популярности Маленкова и новой комбинации вождей.

Либерализация проявлялась не только в этом, она коснулась и репрессивных органов. Публикуется сообщение МВД СССР о ложности обвинений против кремлёвских врачей. Вскоре последвоало официальное осуждение органов госбезопасности, занимавшихся Мингрельским делом.

Поворотный этап формирования нового политического курса --- XX съезд КПСС.

\end{document}