\documentclass{article}
\usepackage[T2A]{fontenc}
\usepackage[utf8]{inputenc}
\usepackage[english,russian]{babel}
\begin{document}
\section{Внешняя политика СССР. Продолжение.}
Наряду со странами, входившими в состав ОВД и СЭВ, были социалистические государства, не зависевшие от СССР. Например, Югославия.

Советский Союз проводил по отношению к Югославии сдержанно-\\доброжелательную политику. Похожие отношения сложились с Румынией. У руководства Румынии находился Чаушеску. Чаушеску был самостоятельный, но в Румынии действовала советская система и СССР её не трогал.

Плохие отношения были с Албанией, которая негативно восприняла десталинизацию в Советском Союзе и увидела угрозу в потеплении между СССР и СФРЮ.

Ещё более жёсткая конфронтация сложилась с Китайской Народной Республикой в результате оттепели. Мао Цзедун посчитал предательством отказ от сталинской модели социализма. Культурная революция в Китае, связанная с политикой жёсткого ограничения и чистки китайской культуры и интеллигенции, также отрицательно сказалась на советско-китайских отношениях. В 1965-1966 гг. из Китая стали уезжать советские граждане, прекратились экономические, культурные и политические контакты. Противостояние дошло до военных столкновений на границе. Такие столкновения произошли на острове Даманский на реке Уссури, а также в средней Азии (Семипалатинская область). Эти стычки не привели к масштабной войне, но вносили напряжённость в отношения двух стран. КНР рассматривалась как потенциальный противник до смерти Мао Цзедуна и Брежнева. Только после этого отношения стали улучшаться. В СССР началась Перестройка, в Китае "--- свои изменения.

Отношения с капиталистическими странами носили противоречивый, но конструктивный характер. Началось потепление с Францией, которую возглавлять Шарль де Голль. Франция вышла из НАТО, это давало Советскому Союзу установить с Францией более тесные экономические связи. Улучшились отношения с ФРГ, поскольку на выборах в Бундестаг победила социал-демократическая партия Германии. Это послужило началом более развёрнутых двухсторонних контактов.

В 1972 году ФРГ и СССР официально заключили мирный договор. Западная Германия признала конфигурацию границ СССР, Польши и ГДР. В 70-е и 80-е советско-западногерманские отношения развивались весьма позитивно.

Наиболее натянутыми были отношения с Великобританией и с Японией. В случае с Великобританией многое зависило от правящей там партии. С консерваторами отношения были натянутые, с лейбористами "--- более тёплые. При лейбористах проблемы были только в тех ситуациях, когда ловился разведчик одной из двух стран.

Советский Союз не заключил с Японией после Второй Мировой из-за того, что Япония требует возвращения четырёх южных Курильских островов.

Из всех капиталистических стран важнейшими являлись отношения с США. Были периоды как конструктивного сотрудничества, так и его крушения. С 1964 года всё открывалось новым обострением: в том же 1964 году, когда Хрущёв был отстранён от власти и началась эпоха застоя, США начали полномасштабные военные действия во Вьетнаме.

В период 1964-1973 гг. американцы пытались поддержать режим южного Вьетнама, а на севере у власти были социалистические силы, поддерживаемые Советским Союзом. В ходе этой войны США потеряли около 60 тысяч человек, а Вьетнам (как северный, так и южный) "--- от одного до трёх миллионов человек. Советские военные специалисты готовили армию северного Вьетнама, советские лётчики непосредственно участвовали в столкновениях и боевых вылетах. Впоследствии американские войска были выведены из Вьетнама на фоне мощной антивоенной кампании в США и других факторов.

Победа вьетнамских коммунистов способствовала поступательному движению социалистов в Лаосе, Кампучии (Камбодже) и других странах региона. В 70-е годы возрастает количество стран Азии и Африки, где приступают к строительству социализма.

В юго-восточной Азии советское руководство поддерживало повстанческие движения, но влияние СССР постепенно падало под давлением Китая. Под влиянием большого количества этнических китайцев в регионе, культурной близости местного населения с Китаем местные движения становились всё более лояльными Китаю. Это обостряло советско-китайские отношения.

Советский Союз пошёл на сближение с США. В 1972 году Никсон приехал в Москву и подписал соглашения об ограничении стратегических вооружений (ОСВ-1). Сокращались межконтинентальные ракеты наземного и подводного базирования. Для обеих стран это было важно, так как экономикам обеих стран было тяжело поддерживать этот потенциал, а последствия конфликта с использованием такого вооружения будут плачевны для обеих стран. Установился режим разрядки.

После визита Никсона часто руководители ездили друг к другу в гости, обсуждали политические и экономические вопросы.

Совещание по сотрудничеству и безопасности в Европе состоялось в 1975 году в г. Хельсинки. На этом совещании были представлены 33 европейских государства, США и Канада. Был подписан заключительный акт совещания, который узаконил послевоенное положение в Европе и в мире. Выражалось намерение, что никто не будет навязывать возможное изменение этого строя. Признавались установившиеся в Европе и в мире границы.

Советский Союз и США продолжали пытаться остановить гонку вооружений. В 1979 году был подписан договор ОСВ-2, в котором предполагалось ещё установить некие ограничения на размещение и производство ядерного оружия. Этот договор силы не получил, так как в декабре 1979 года Советский Союз ввёл войска в Афганистан. Сенат США не ратифицировал договор ОСВ-2. Снова вспыхнула гонка вооружений. В западной Европе размещаются американские ракеты средней дальности, оснащённые ядерной боеголовкой.

\subsection{Афганская война и её хронология. 1979-1989 гг.}
Причины Афганской войны (ввода советских войск на территорию Афганистана) в следующем:

В 1978 году в Афганистане произошла революция. К власти пришла народно-демократическая партия Афганистана. Она вводила Афганистан в русло советской внешней политики. Против народно-демократической партии выступили внутренние силы, началась гражданская война. Афганские коммунисты обратились за помощью к СССР, тот ответил финансовой поддержке.

В Иране тем временем происходит исламская революция. Новый аятолла Хоммейни запретил марксистские партии. А Китай ввёл войска во Вьетнам. Это показало, что Советский Союз теряет позиции в крупных азиатских странах. Именно под влиянием этих причин руководство страны во главе с Брежневым принимает решение отправить советское войска. Решение было принято Брежневым и несколькими первыми фигурами. Против выступил только Косыгин, который знал, что у государства есть серьёзные экономические проблемы, которые не позволят успешно вести войну.

С декабря 1979 года вводится ограниченный контингент советских войск в Афганистан для выполнения интернационального долга, помощи братскому народу Афганистана. Помощь моджахедам "--- противникам народно-демократической партии Афганистана "--- оказывали США.

Почти в полном составе Генеральной ассамблеей ООН принимается резолюция, осуждающая ввод войск в Афганистан. Эти события отрицательно сказались на внешнеполитической репутации СССР.

На первом этапе войны С декабря 1979 года по февраль 1980 года советские войска вводятся в Афганистан и размещаются погарнизонно в разных частях страны.

Второй этап. Март 1980 --- апрель 1985 года. Ведутся активные боевые действия. Советский Союз проводит рекомплектацию и укрепление союзных сил.

Третий этап. Апрель 1985 года --- янваль 1987 года. Наступает перелом в войне. Новое советское руководство предпочитает снизить активность в Афганистане. Местные войска поддерживаются авиацией и сапёрами. Некоторые соединения выводятся из страны.

Заключительный этап. Январь 1987 года --- февраль 1989 года. Подготовка советских войск к выводу и их полный вывод из Афганистана.

Потери Советского Союза подсчитать сложно. Обычно они колеблются от 15 до 26 тысяч человек. Среди афганцев потери гораздо больше: от 600 тысяч до 2 миллионов человек.

Афганская война привела к подрыву и без того проблемной экономики Советского Союза. Обществу не была понятна идеологическая и конкретная цель этой войны.

В начале 1980х отношения со странами Запада и США обострились настолько, что эти годы можно называть пиком Холодной войны. В эти годы конструктивные контакты практически прекратились. Вводится эмбарго на поставку зерна в СССР.

В средствах массовой пропаганды стран обоих блоков развиваются мысли о подготовке к войне. Под прикрытием разрядки ведётся противостояние в странах Третьего мира.

С 1985 года опять начинается потепление и разрядка.

\section{Советский Союз в годы перестройки. Внутренняя политика и экономика.}
Кризисные явления в жизни общества отчётливо выявились и стали стремительно возрастать. Положение Советского Союза отличалось неопределённостью: СССР претендовал на статус великой державы, но его было всё труднее удерживать. Советский Союз занимал $\frac{1}{6}$ часть суши. По последней переписи население СССР достигло 287 миллионов человек. В городах проживало $67\%$ населения, причём рост происходил за счёт неславянского населения "--- средней Азии, Закавказья и пр. В российских регионах (включая Белоруссию, Украину и Прибалтику) этот рост был не таким быстрым.

Большую долю ресурсов поглощал огромный военно-промышленный комплекс СССР.

Несмотря на определённые достижения экономического и социального состояния, Советский Союз находился в глубоком предкризисном состоянии. Советская промышленность буксовала и это выглядело ещё более неудобно на фоне ускоряющегося развития стран Запада.

Идея Ленина о мировой революции зашла в тупик. Перед правящими кругами страны встала задача сохранения своей власти в этих условиях. При Андропове проводились чистки, подтягивалась дисциплина, но всё это оказалось безрезультатным. Нарастает идея глубокой реформации общественной жизни в СССР.

Кто такой Горбачёв? Родился в 1931 году в крестьянской семье в Ставропольском крае (на тот момен это была Северковказская область). Показал себя как отличный механизатор. В 18 лет получил орден трудового красного знамени. Чуть позже стал кандидатом в члены КПСС, хотя обычно в этом возрасте только в комсомол вступали. Такие успехи позволили Михаилу Горбачёва поступить в Московский Государственный Университет на юридический факультет. Окончил его и в 1952 году стал членом партии. С этого времени начинается работа в комсомольских и партийных структурах. В 1971 году стал членом секритариата ЦК КПСС, а в 1980 году "--- членом политбюро ЦК КПСС. Горбачёв оказался на вершине государственного управления в молодом возрасте.

В 1985 году после смерти Черненко Горбачёв избран генеральным секретарём ЦК КПСС. Председателем совета министров стал Рыжков. Не было даже определённо сформулированных целей.

Собирается пленум (общее собрание членов и кандидатов) ЦК КПСС, на котором Горбачёв делает доклад и сообщает о планах реформ, направленных на ускорение развития страны. Выдвигается задача достичь качественно нового состояния советского общества. Нужно модернизировать производство и достичь мирового уровня производительности труда в материальной и духовной жизни людей.

В самой перестройке можно выделить два крупных этапа.

1. 1985-1988
2. 1989-1991

\subsection{Первый этап. 1985-1988}
Изменений практически не произошло. Советский режим продолжает функционировать на основе Конституции 1977 года. Председатель президиума Верховного совета "--- Громыко, потом Горбачёв.

Фактическую власть вершили исполкомы советов.

В этот период много говорят о демократизации и гласности "--- более широком и свободном обсуждении внутренних проблем.

\subsection{Второй этап.}
Верховный совет принял изменения и дополнения к Конституции 1977 года. Главные изменения касались избирательной системы и принципа функционирования выборных органов. Выборы становятся альтернативными и состязательными. Весной 1989 года проводятся выборы народных депутатов СССР, а через год "--- народные депутаты союзных, автономных республик и местных советов.

Над Верховным советом появился съезд народных депутатов (аналог съезда советов из ранних этапов истории СССР). Депутаты выбираются весной 1989 года. Выборы показали неготовность населения к альтернативным выборам. Были кандидаты не только от местных органов, но и от общественных организаций.

Состоялся первый съезд народных депутатов СССР с мая по июнь 1989 года. Когда народные депутаты собрались, то выяснилось, что на  съезде два блока "--- блок коммунистов и блок демократов. Председателем верхового совета СССР.

Председателем совета министров стал Рыжков. Съезд утвердил и другие важнейшие должности "--- председателя Верховного суда и генерального прокурора. Съезд рассмотрел основные направления внешней и внутренней политики СССР. В дальнейшем состоялось всего 5 съездов народных депутатов. Они принимали изменения в Коснтитуцию, в избирательную систему\dots

В сфере исполнительной власти был введён институт президентства. Президентом был избран Горбачёв по решению чрезвычайного съезда народных депутатов. В этом смысле Советский Союз копировал американскую модель. В феврале 1991 года вице-президентом СССР стал Янаев.

Деятельность депутатов продолжалась до сентября 1991 года, но в августе 1991 года случился пиздец и после попытки госпереворота народных депутатов стали обвинять либо в пособничестве ГКЧП, либо в бездействии. В итоге народные депутаты объявили о самороспуске. В сентябре 1991 года их деятельность прекратилась.

Параллельно с общесоюзными перестраивались органы власти в республиках примерно те же изменения. В некоторых республиках к власти приходили оппозиционеры и, в частности, националисты. Большинство союзных республик объявляют о суверенитете и претендуют на независимость. Наиболее активными эти события были в Прибалтике. Кроме того в Молдавии и на южном Кавказе шли такие процессы. После выборов в верховные советы этих республик победили оппозиционеры. В центральной Азии победили сторонники СССР.

В русских республиках не было преимущества ни у КПСС, ни у оппозиционных сил. Это привело к затруднению работы законодательных органов в России, на Украине и в Белоруссии. При низком уровне политической культуры решение даже незначительных задач вызывало большую полемику.

\end{document}