\documentclass{article}
\usepackage[T2A]{fontenc}
\usepackage[utf8]{inputenc}
\usepackage[english,russian]{babel}
\begin{document}

\section{СССР в 1964-1985 гг.: внутренняя политика, экономика}
\subsection{Политика}
Указанное 21-летие называют периодом наиболее стабильного развития советского общества, эпохой застоя. Официальная пропагана гласила, что СССР находится в состоянии развитого социализма. Обещания Хрущёва построить коммунизм подзабывались.

В этот период назревал глубокий кризис советского общества.

В этот период наблюдался значительный демографический рост. Население в 1970 состовляло 240 млн человек, а через 15 лет --- 280 миллионов.

Сохраняется всевластие партийно-государственной номенклатуры (правящая элита общества "--- партийный аппарат).

В 1977 году принимается Конституция СССР. Советы депутатов трудящихся переименовываются в советы народных депутатов. Появилась статья №6, юридически закрепившая положение КПСС. Предложения исполкомов обязательно утверждались советами депутатов.

Население становится политически пассивным.

В 1965 году председателем президиума Верховного совета становится Подгорный. Первый секретарь ЦК "--- Леонид Ильич Брежнев. Когда Подгорного отправили на пенсию, Брежнев возглавил ещё и президиум. Косыгин возглавил совет министров СССР. Кроме этих трёх в Политбюро входили ещё несколько человек. Наиболее влиятельное лицо "--- Брежнев.

Брежнев родился в Екатеринославской губернии в семье рабочих. Окончил курсы и институт в городе Каменском, после чего стал инженером. В 1931 году вступил в партию, в конце 1930-х годов перешёл на партийные должности и делал быструю карьеру. Участвовал в Великой Отечественной Войне на политических должностях. Бывал и в боевых ситуациях. Участвовал в освобождении Новороссийска. Окончил войну в звании генерал-майора. После войны находился на руководящих партийных должностях в разных республиках. В 1956 году стал одним из секретарей ЦК партии. В 1960-х стал председателем президиума Верховного совета. Участвовал в заговоре против Хрущёва и занял место первого секретаря ЦК партии. В 1966 году стал вторым после Сталина генеральным секретарём ЦК партии. Четырежды Герой СССР и герой социалистического труда. В 1976 году получил звание маршала СССР. В этом же году пережил клиническую смерть и был тяжело больным человеком.

Конституция отменила совнаркомы, восстанавливаются вертикальные связи в экономике. Возрастает роль бюрократии, усиливается нормативность, становится больше бумажной волокиты. Отличительна особенность --- разбухание бюрократии. Советская бюрократия стала самой большой в мире, уступая только Китаю. Советская бюрократия достигла 18 миллионов человек (1/7 часть трудоспособного населения). Общество испытывало дефицит средств для содержания бюрократического аппарата.

Темпы роста экономики резко снизились. Реальная производительность труда не увеличивалась, особенно в сельском хозяйстве. Эксперименты показали свою неэффективность.

Власть высших партийных должностей становится всё больше и очевидней. Выборы в партийные органы представляют собой утверждение единственного кандидата.

Это ещё более влияло на пассивность населения.

Подобная бюрократизация не исключала социальные лифты, но продвижение в любой сфере становилось возможным только для активных членов КПСС, полностью лояльных вышестоящему руководству.

Высшее руководство партии превратилось в элиту --- людей, обладающих всеми привелегиями. Они обладали правом выезда из страны, имели доступ к закрытым спецраспределителям, пользовались государственными дачами и автомобилями.

XXIII съезд КПСС положил конец разоблачению сталинизма. Стали встречаться воспоминания о Сталине в положительном контексте. Постепенно вырос культ личности Брежнева. В адрес личности Брежнева постоянно звучали словословия. Пик культа личности Брежнева был достигнут ближе к его смерти в 80х годах. Этот новый культ, в отличие от сталинского, всё больше наблюдал фарс.

Брежнев был тяжелобольным человеком и не мог работать больше двух-трёх часов в день. Дееспособность Брежнева явно падала, отчего появилось огромное количество анекдотов Брежнева, подчёркивающих не вполне дееспособное состояние лидера страны. Не пользовались уважением и другие деятели партийного руководства, которое тоже представлял из себя престарелых и больных людей.

Последнюю попытку оживить административный аппарат предпринял Юрий Владимирович Андропов, пришедший к власти в 68 лет. Прежде он возглавлял КГБ СССР. Он попытался придать бюрократическому аппарату реальную эффективность. Увеличилась роль милиции. Андропов не добился серьёзных результатов. В феврале 1984 года Андропов умер, не завершив чистку аппарата. Её продолжил Черненко. К 1985 году Политбюро естественным образом уходило из жизни.

\subsection{Экономика}
Прежде всего власти хотели изменить положение в сельском хозяйстве путём повышения материальной заинтересованности колхозников.

Что предприняло правительство Косыгина? Снижается план обязательных закупок зерна и объявляется неизменным на 10 лет. То, что произовдится сверх плана, государство закупает по повышенным ценам. Снимаются ограничения с личных подсобных хозяйств.

Власти пытаются повысить самостоятельность хозрасчёта на советских предприятиях. Каким образом? Восстановление отраслевого принципа, сокращение числа плановых показателей с 30 до 9. Сохранение в распоряжении предприятий большей доли прибыли и повышение оптовых цен на реализованную продукцию стимулировали производство экономическими рычагами.

Это вызвало отрицательную реакцию со стороны партийно-бюрократического аппарата: интересы предприятий и руководства страны стали расходиться. От этой реформы практически полностью отказались и пошёл откат к старой системе.

В 1974 году всем советским колхозникам были выданы паспорта. Открылась возможность более свободного переселения в города.

Экономика страны характеризовалась перманентным снижением темпов роста. Реформы Косыгина задержали этот процесс, но когда от реформ отказаались, всё стало ещё хуже. Официальная статистика не могла скрывать это падение. Основнаяя доля экспорта "--- нефть.

В западной Сибири идёт разработка огромных месторождений нефти и газа. Очень удачно начался рост цен на мировом рынке. Советский союз получал от нефтедолларов огромные доходы.

Негативные явления на производстве возрастали, в частности, в плане уравниловки, что лишало граждан стимула развиваться.

В промышленности ручном трудом было занято 40\% всех работников. 

Негативные процессы стали отражаться в социальной сфере. Медленно шло жилищное строительство. Всё чаще возникали проблемы с транспортом, продовольствием и медицинским обслуживанием. Заработная плата росла незначительно и только у низкоквалифицированного труда. Граждане с трудом удовлетворяли свои потребности в товарах широкого потребления.

Реальный уровень жизни постепенно падал. Возникал чёрный рынок. В обществе нарастают негативные тенденции к власти. Власти пытаются удерживать контроль, ответом на это стало диссидентское движение. Как правило, внутри страны диссиденты оказывались в тюрьмы. Иногда их высылали за границу.

Выступление диссдентов в 1965 году связано с делом Синявского и Даниэля, опубликовавших сови произведения за рубежом. Их привлекли за эту публикацию. Были осуждены на 7 и 5 лет лагерей соответственно.

Музыкант Расстрапонович был выслан из страны. 

\section{Внешняя политика СССР в 1964-1985 гг.: внешняя политика}
Реальная угроза ядерной катастрофы требовала снижения напряжённости и сближения Советского Союза и США. С другой стороны, обе страны намеревались стать мировым гегемоном. Для Советского Союза эта цель отодвигалась. Власти СССР надеялись, что социализм завоюет весь мир, но начнётся это завоевание с колоний, вышедших из-под контроля. Исходя из этой доктрины, советская политика развивалась по трём направлениям:
\\
\begin{enumerate}
    \item Социалистические страны "--- кореша
    \item Капиталистические страны "--- чушпаны
    \item Страны третьего мира "--- школота, могут стать корешами
\end{enumerate}

К власти во многих странах стали постепенно приходить более либеральные силы, приспосабливающие свои модели социализма под современные реалии и проводили реформы, не соответствующие традиционным представлениям о социализме. Совет экономической взаимопомощи играет всё меньшую роль, потому что социалистические страны пытаются наладить отношения с капиталистическими странами из меркантильных причин.

Советский союз определял основные направления действия своих союзников. В 1968 году, чтобы остановить либерализацию СССР в Чехословакии, Советский Союз и другие страны Варшавского договора (Польша, Венгрия) вводят свои войска.

Лидеры соц. стран ездят в Москву и в Крым для консультации. Советский Союз оказывает экономическую поддержку в рамках СЭВ. В 1971 году была принята программа укрепления сотрудничества, предполагающая обеспечение союзников дешёвыми ресурсами. Нефтепровод "Дружба" снабжал Польшу и Германию; Чехословакию, Австрию и Венгрию по ценам, значительно меньше мировых.

В 70-е годы проводились общие военные манёвры на территории СССР, Польши и ГДР.

Не менее показательны события в Польше в начале 80х. Был создан независимый профсоюз "Солидарность" и началось массовое забастовочное движение против социалистического режима Польской народной республики. Советский Союз оказывал разнообразную помощь. В Польше вводится военное положение, протестное движение подавлено. Подготовленный ввод советских войск в Польшу не состоялся.
\end{document}