\documentclass{article}

\usepackage[T2A]{fontenc}
\usepackage[utf8]{inputenc}
\usepackage[english,russian]{babel}

\begin{document}
\section{Политическое развитие страны в период оттепели}
Всю власть в своих руках сконцентрировал Н. С. Хрущёв
\subsection{Биография Хрущёва}
Родился в 1894 году в семье шахтёра в Курской области. Вступил в 1917 году в партию и в 1918 году обучался на рабфаке. В 1920е поступил в промышленную академию в Москве и познакомился с Надеждой Алилуевой.

В 20-30е годы он активно делает партийную карьеру, участвует в чистках и репрессиях.

В 1938 году был первым секретарём ЦК КП Украины, в 1939 вошёл в политбюро ЦК КПСС.

После Сталина пост генсека был переименован в пост первого секретаря партии. Именно этот пост занял Хрущёв.

\subsection{Доклад Хрущёва}
Коренной перелом произошёл на XX съезле КПСС в феврале 1956 года. Хрущёв сделал доклад "О культе личности и его последствиях". Хрущёв разоблачил преступный характер деятельности Сталина, все пороки государственного аппарата были списаны на Сталина и его ближайшее окружение. Всячески подчёркивалось, что культ личности не изменил социалистический курс государства. Поскольку наиболее одиозные фигуры сталинской власти были устранены, это снимало ответственность с других лидеров партии. Оставшееся руководство оказывалось вне критики.

С этого момента начинаются более серьёзные движения к либерализации. Процесс реабилитации жертв сталинского режима начался до  XX съезда, но после он пошёл ещё быстрее. С 1956 по 1961 год было реабилитировано 700 тысяч человек. Многие реабилитировались посмертно. Восстанавливали доброе имя тех, кто был репрессирован ещё в 20е и 30е годы. Но процесс реабилитации не коснулся Троцкого, Бухарина, Каменева, Рыкова. Снимаются обвинения с депортированных народов.

Этот процесс продолжался до начала 60х годов, далее волна реабилитаций стала ослабевать.

В это же время происходят изменения в уголовном законодательстве. Статья за антисоветскую деятельность сохраняется, но отменено понятие врага народа, возраст уголовной ответственности повышен с 14 до 16 лет. Насилие во время следствия запрещено, рассмотрение обвинения без присутствия обвиняемого запрещается, обвиняемому предоставляется адвокат.

\subsection{Строительство социализма}
На XXII съезде в 1961 году Хрущёв объявил курс на коммунизм.

Что такое коммунизм? Цель коммунизма --- превысить экономический уровень капиталистических стран, заняв первое место по продукции на душу населения, ликвидировать классовые различия и создать однородное общество, ликвидировать различия между образом жизни города и деревни, ликвидация различия между физическим и умственным трудом, возрастание общности наций, развитие черт человека коммунистического общества (высокая идейность, широкая образованность, моральная чистота, физическое совершенство), управление общественными делами на основе широкой социалистической демократии. Предполагалось достичь это к 1980 году.

Ради достижения этих результатов Хрущёв планирует обновление внутрипартийных кадров и расширения внутрипартийной демократии, что вызвало сопротивление партийного аппарата: он привык к принятию решений сверху вниз и ополчился на Хрущёва. К началу 1960х годов были и трудности экономического характера. Эти трудности были поставлены в упрёк Хрущёву и создалась благоприятная атмосфера для переворота.

Противники Хрущёва подготовили заговор, обвинили Хрущёва в валинтаризме и субъективизме. Путём голосвания ЦК Хрущёв был лишён должности первого секретаря и других должностей и отправлен на пенсию.

Во главе советского государства триумвират Косыгина (совет министров), Подгорного (президиум), Леонид Ильич Брежнев (генеральный секретарь).

\subsection{Внешняя политика}
С одной стороны увеличиваются контакты с западными странами, обсуждается сокращение вооружений. Нормализуются отношения с Югославией. Хрущёв восстановил контакты с Югославией, подружился с Тито. Хрущёв достиг компромисса по Корее, заключив перемирие в 1953 году, означавшее распад Кореи на Корейскую Народно-Демократическую Республику и Республику Корею.

Решена судьба Австрии, также оккупированной войсками союзников. Хрущёвское руководство дало согласие на воссоздание единой независимой Австрии. Австрия с этого момента развивалась на западной основе.

Советское руководство препятствовало процессам либерализации в\\ странах-союзницах. В ГДР в 1953 году и в Венгрии в ответ на выступления местного населения вводятся войска.

В 1955 году заключается Варшавский договор --- военно-политический блок, который должен противостоять НАТО. В него входит СССР и 7 стран восточной Европы --- Польша, Чехословакия, Венгрия, Румыния, Болгария, ГДР и Албания.

Главное направление внешней политики СССР --- отношения с США. Обе стороны наращивают ядерный потенциал. В 1957 году СССР впервые испытывает межконтинентальные баллистические ракеты, что серьёзно меняет расклад сил.

США пытаются разместить вооружения как можно ближе к границам СССР.

В 1962 году разразился Карибский кризис. В ответ на размещение американских ракет в Турции СССР разместил ракеты на территории Кубы. США попытались остановить доставку ракет на Кубу. Встала угроза ядерной войны. В результате личных переговоров Кеннеди и Хрущёва по различным каналам было достигнуто соглашение: и СССР, и США убирают ракеты от границ друг дргуа.

Напряжённость в отношениях постоянно то возрастала, то ослабевала. В 1961 году был сбит американский самолёт-разведчик, пилот был захвачен в плен.

\section{Социально-экономическое и культурное развитие страны в период оттепели}
В районах работали советы народного хозяйства. Каждая из союзных республик делилась на много таких совнархозов. Они подчинялись республиканским правительствам. Считалось, что децентрализация приблизит руководство к низовым звеньям --- предприятиям --- и это принесёт видные экономические результаты. Но этого не произошло.

Происходил экономический подъём. В V пятилетке рост промышленного производства составил 70\%. На востоке страны, в Прибалтике, Поволжье, на Кавказе создавались целые комплексы предприятий. Совершенствовалась система оплаты труда.

Акцент делался по-прежнему на тяжёлую промышленность, рост товаров широкого потребления отставал. Велись работы по созданию единой энергетической системы в европейской части страны. Создаётся множество новых электростанций. Создаётся волжско-камский каскад гидроэлектростанций. Переоценить значение этих электростанций для экономики невозможно. Позитивное значение этих строек далеко. Создание волжско-камского каскада привело к серьёзным социальным изменениям: затапливались большие территории, с которых приходилось выселять население. Отвратительно сказывается на экологию. Создаётся первая АЭС.

VI пятилетка превысила ожидаемый рост. После неё в 1959-1965 гг. СССР жил в рамках семилетнего плана с показателями выше, чем раньше. Однако семилетка оказалась авантюрой, что было признано после отставки Хрущёва. План семилетки не был достигнут, но за это время вступило в строй 5,5 тысяч предприятий, удвоилась мощность электростанций, ускорилась добыча ресурсов. Изменился топливный баланс страны. Нефть и газ повысили своё участие в экономике с 30\% до 50\%.

Увеличилась протяжённость и пропускная способность железных дорог. Прекращается производство паровозов, вместо них вводятся в строй тепловозы и электровозы. Развивается воздушный транспорт.

Ещё одно прорывное событие 1957 года --- спуск первого в мире атомного ледокола "Ленин", что важно для освоения Арктики.

К этому времени относят начало научно-технической революции в Советском Союзе. Наука становится ведущим фактором производства, в производстве с этого времени всё решают автоматизация, научно-технический контроль, управление производственными процессами через электронику, создание новых конструкционных материалов. К концу 50х годов СССР явно опережает конкурентов других стран в освоении космоса, в химизации народного хозяйства, в развитии электроники. Всё это подкрепляется крупными государственными программами.

Технический прогресс повлиял как на гражданскую, так и на военную технику. Испытан первый пассажирский реактивный самолёт Ту-104. Осуществлён запуск многоступенчатой межконтинентальной баллистической ракеты. Запущен первый в мире искуственный спутник Земли. СССР становится лидером в освоении космоса.

Королёв --- руководитель советской космической программы --- благодаря которому состоялся полёт Гагарина.

В 50е годы распространилось телевидение. В стране действовало больше 50 телецентров, но телевизор пока ещё был доступен не всем: телеприёмников было чуть более трёх миллионов. В каждый дом телевизор пришёл в конце 70х годов.

Некоторые отрасли хозяйства не справлялись с плановыми заданиями. Особенно отставали сельское хозяйство и лёгкая промышленность. Хрущёв выступал против кредитования лёгкой индустрии, хотя это в своё время предлагал Маленков.

В 1953 году ЦК КПСС выработал новый курс на развитие сельского хозяйства. Большинство советских людей всё ещё проживали в сельской местности. С 1953 года задумались о повышении материальной заинтересованности колхозников. Повышались закупочные цены, понижались объёмы обязательных закупок, увеличиваются вложения в сельскохозяйственное машиностроение в 1,5-2 раза.

Финансовые вливания в сельское хозяйство не вносили изменения в землепользования. Опека государственных органов демотивировала крестьян.

Помимо финансовой помощи решать сельскохозяйственные проблемы пытались через увеличение посевных площадей, незначительное увеличение экономической самостоятельности и др.

В рамках экстенсивного пути развития через увеличение посевных площадей разворачивается программа освоения целинных земель Казахстана, юга Сибири и Заволжья. Их начинают распахивать с 1954 года и это дало кратковременный позитивный результат. Дело в том, что в степных регионах мощность гумуса очень маленький. Начинается эрозия почв. На таких полях начинают быстрее расти овраги, образуются солончаки, начинаются пыльные бури.

В 1957 году Хрущёв выдвинул лозунг "Догнать и перегнать Америку" по производству всего, прежде всего по продуктам питания. Он предложил утроить производство мяса в стране.

Секретарь рязанского обкома КПСС Ларионов, получив информацию, решил утроить производство мяса за год. В рамках рязанского почина стали забивать на мясо племенной скот и молокодающих коров, забирать личный скот колхозников. Этого не хватало, поэтому стали закупать скот для забоя. Рязанская область выполнила план по мясу. План по молоку был провален. В 1960 году резко упало производство мясо, молока и зерна: колхозники отказались обрабатывать колхозные земли. В 1959 году Ларионов получил звезду героя социалистического труда, в 1960 году Ларионов умер от остановки сердца. Пошли слухи про самоубийство. Возможно, Ларионов умер от того, что он переживал за свою ошибку.

СССР не смог обогнать США по производству мяса и молока до сих пор.

Кукурузная кампания. В 1959 году Хрущёв посетил США, встретился там с президентом. Съездил по некоторым штатам и посмотрел достижения американского народного хозяйства. Особенно Хрущёв впечатлился в штате Айдахо. Под впечатлением от этой поездки Хрущёв стал наращивать на увеличении выращивания именно кукурузы, ведь она может использоваться и как пищевая культура (из неё можно делать муку, кукурузные хлопья, варёную кукурузу и многие другие продукты), как кормовая культура. Но кукуруза --- тепло- и влаголюбивое растение. Хотя миллионы гектар засаживались кукурузой, местами это приводило к потерям.

Хрущёв занялся реформой МТС. Эффективность техники снизилась.

Правительство развернуло кампанию по борьбе с личным хозяйством. Это привело к двояким последствиям. С одной стороны, поголовье государственного скота выросло. С другой строны, оно ударило по способности крестьян обеспечивать себя.

Рост производства сельского хозяйства был незначительным, а потом стал снижаться. Причины следующие:
\begin{itemize}
    \item Неэффективное использование ресурсов
    \item Грубое админитсративное вмешательство
    \item Уменьшение капиталовложений
    \item Лишение крестьян собственных земель
\end{itemize}

С 1963 года Советский Союз начинает закупать крупные партии зерновых у капиталистических конкурентов --- Канады, США, Австралии и других стран.

Такие трудности заставили власти начать самую непопулярную меру --- повышение цен на продукты питания. Иногда население стало доводить своё недовольство весьма радикальными способами. В 1962 году в Новочеркасске трудящиеся решили устроить демонстрацию с призывами улучшить уровень жизни. С помощью войск и милиции демонстрация была разогнана, зачинщиков приговорили к расстрелу. Кроме Новочеркасска крупных выступлений не происходило нигде, уровень жизни большинства людей повышался. В городах уровень жизни рос быстрее и был заметнее. Увеличивались зарплаты и пенсии. В 1964 году начали платить пенсии колхозников. Трудящиеся стали получать отдельные квартиры. К 1963 году для всех рабочих и служащих установлен семичасовой рабочий день при шестидневной рабочей неделе. Позже вернут восьмичасовой рабочий день при пятидневной рабочей недели. Выросли зарплаты.

Жилищное строительство развивалось. С 1957 года государство вкладывает куа больше средств в строительство жилых домов. Стали сооружаться микрорайоны из пятиэтажных панельных и кирпичных типовых домов низкой себестоимости. Вводились миллионы квадратных метров жилых площадей.

Откуда слово оттепель? Известный писатель Илья Эринбург применил это слово по отношению к этой эпохе.

Оттпель проявляла себя в литературе и искусстве. Появляются новые направления и жанры. Первые публикации такого рода происходят в журнале "Новый мир". Потом десталинизация усиливается, но не отказ от ленинского курса. Н. С. Хрущёв часто лично встречался с писателями, представителями культуры и пр. Известны истории, когда литераторы за излишне свободное творчество наказывались. Так, от печати романа Пастернака "Доктор Живаго" отказались, но его произведение обрело популярность в западных странах. Пастернак был исключён из союза писателей.

\end{document}