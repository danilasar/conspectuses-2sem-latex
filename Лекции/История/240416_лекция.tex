\documentclass{article}

\usepackage[T2A]{fontenc}
\usepackage[utf8]{inputenc}
\usepackage[english,russian]{babel}
\usepackage{mathtools}
\title{Название}
\author{Даня Грозный}
%\with{Владимир Владимирович}

\begin{document}
С 23 апреля история переносится на 27 апреля.

Весной 1943 года наступила стратегическая пауза и стороны наращивали ресурсы для решающего сражения. СССР наращивал вооружение и принимал новые образцы. В частности, тяжёлый танк КВ и средний танк Т-34. Появились истребители Ларг-3, Як-1 и Ил-2.

В составе советских войск образуются иностранные части. Например, Чехословацкий батальон, польская дивизия имени Тадеуша Костюшко, французский авиаполк "Нормандия-Неман". Были и югославские части.

Германия и её сателлиты провели тотальную мобилизацию. Соотношение сил в июле 1943 года уже складывалось в пользу советских войск. Превосходство по числу техники было кратным.

\subsection{Сближение с Церковью}

22 июня 1941 года митрополит Сергий выступил с обращением к верующим, призвав защитить Отечество от врага. Стали собираться средства на военные нужды, была создана танковая колонна имени Дмитрия Донского. Коллаборационисты отлучались от Церкви. Сталин разрешил избрать нового Патриарха, им стал митрополит Сергий. Был образован Священный синод, открыто несколько семинарий по подготовке священников. Веротерпимость не отменяла контроля над Церковью.

\subsection{Немецкий оккупационный режим}

Возникает оккупационная система управления. Было создано министерство оккупированных восточных территорий, ему подчинялось несколько рейхскомиссариатов --- Остланд (Прибалтика), Украина. Молдавия оказалась под управлением Румынии. Часть Карело-Финской ССР была оккупирована Финляндией.

Нацисты развивали коллаборационистское движение. Советские органы власти закрывались, общественные организации запрещались. В селе создавались комендатуры, в которых приглашали пронемецки настроенных селян. На селе функционировала полиция, в городах --- отряды СС. Они контролировали захваченные территории и осуществляли карательные акции против противников режима.

Вёлся строгий учёт местного населения, подлежавшего регистрации в полиции. Для местного режима вводилась трудовая повинность, невыполнение которой жестоко каралось. Местное население обеспечивало немецкую армию пищей. В Германию и некоторые другие страны было принудительно перевезено около 5 млн человек.

Коммунисты, евреи и цыгане отправлялись в концентрационные лагеря, где осуществлялись массовые убийства людей всех возрастов и любого пола.

На это советский народ отвечал борьбой в тылу врага. В течение 1943 года на оккупированных территориях сражалось до 250 тысяч человек. Партизаны проводили диверсионные операции. Наиболее известные --- рельсовая война и "Концерн". Партизаны устроили до 250 тысяч крушений поездов, взорвали множество мостов и часть военной техники противника.

\subsection{Освобождённые районы}

На освобождённых районах восстанавливалась экономика

\subsection{Курская бритва (5 июля -- 23 августа 1943 года)}

Образовался Курский выступ --- дуга, про которой проходил фронт. Немецкое командование решило провести стратегическую операцию "Цитадель". Планировалось, согласно ей, разгромить советские войска, прорвав оборону с юга и севера. Добившись здесь успеха, развернуть наступление на Москву с юга. Для этого было сосредоточено около 50 дивизий численностью до 900 тысяч человек. Здесь было большое скопление техники противника.

Им противостояли войска Центрального и Воронежского фронтов, имеющие превосходство в танках, самоходных орудиях и даже самолётах. В тылу стоял Степной фронт, готовый поддержать оборону в тяжёлые моменты.

Ставка решила сначала обороняться в первую очередь до разгрома танковых и механизированных сил врага, а затем перейти в контрнаступление. Советская армия подготовила эшелонированную линию обороны глубиной до 300 километров.

Около недели советские войска упорно оборонялись. Немцы наступали с юга и с севера от Курска. Продолжались ожесточённые и кровавые бои с огромными потерями с обеих сторон. К 12 июля наступление выдохлось и советские войска начали наступать на Орловском и Белгородско-Харьковском направлениях.

Победа на Огненной дуге стала коренным переломом в ходе Великой Отечественной Войны. После победы под Курском советская армия удерживала инициативу до конца войны. Это способствовало и успеху союзников.

После поражения под Курском немецкое командование пыталось перевести войну на позиционные формы, но  войска Юго-западного и Южного фронтов освобождают Донбасс и к концу сентября выходят к Днепру.

Здесь 4 советских фронта переименовывают в 4 украинских фронта. С тяжёлыми боями они успешно форсируют Днепр. 6 ноября советские войска вступили в Киев. Освобождаются Запорожье и Днепропетровск, противник блокируется на Крымском полуострове.

В это же время на севере действуют Калининский, Западный и Брянский фронты, которые отбрасывают врага на  300 километров от Москвы. Начинается освобождение Белоруссии.

Победы советских войск позволяют в корне переменить обстановку в северной Африке. Африка постепенно очищается от немецко-итальянских войск. Начинается вторжение в Италию. Американцы и Британцы осуществляют десант сначала на Сицилии, а потом и на Аппенинском полуострове. Это помогло свергнуть режим Муссолини. Новое итальянское правительство объявляет войну Германии и заключает мир с союзниками.

Активизировалось дипломатическое и военно-политическое сотрудничество США и Великобритании. Состоялась Тегеранская конференция (28 ноября -- 1 декабря 1943 г.). Здесь заседала большая тройка в лице Сталина, Черчилля и Рузвельта. Операция "Оверлоад" была намечена на май 1944 года. Обсуждались контуры послевоенного устройства мира. Черчилль и Рузвельт высказали мысль о разделе Германии, Сталин был скептически настроен к этой идее. Он боялся, что резкое ослабление Германии оставит СССР один-на-один с капиталистическими державами. Иосиф Виссарионович надеялся создать дружественную силу в Европе.

\section{Завершающий период Великой Отечественной Войны (декабрь 1943 -- май 1945 гг.). Окончание Второй мировой. Итоги войн для СССР}

Важнейшие тенденции завершающего периода ВОВ:
\begin{itemize}
    \item Возрастающая мощь стран Антигитлеровской коалиции. Им удалось в несколько раз превысить возможности Германии и её союзников
    \item Резкое ухудшение возможностей Германии. Истощаются материальные и людские ресурсы. Командование Вермахта перешло к жёсткой позиционной борьбе: крепости, ДОТы\dots
\end{itemize}

Однако враг был по-прежнему силён. По людским ресурсам превосходство было незначительно, по танкам и штурмовым орудиям был паритет, по самолётам СССР превосходил Германию в несколько раз. Это связано с ростом производства военной техники в 4-8 раз и помощью союзников (ленд-лизом: lend, а не land -- займы). Ленд-лиз --- название закона, принятого в США ещё в 1941 году. Он предусматривал поставки боеприпасов, техники, продовольствия, медикаментов, стратегического сырья без предварительной оплаты. На условиях беспроцентных долгосрочных кредитов вооружение должно было быть оплачено после завершения войны.

СССР получил более 18 тысяч самолётов, 18,5 тыс. ед. бронетехники, 400 тыс. грузовых и легковых автомобилей, локомотивы и другие виды военных предметов и стратегических материалов. Объём ленд-лиза превысил 10 миллиардов долларов.

Советский Союз мобилизовал все возможные ресурсы. Но был ещё силовой (репрессивный) ресурс, который тоже использовался. Как и в других странах, Советским Союзом практиковались массовые принудительные переселения: переселялись поляки и другие не принявшие новую власть представители титульных наций территорий, присоединённых к СССР. Все они отправлялись в отдалённые регионы Советского Союза.

В западных регионах СССР развернулось мощное коллаборационистское движение. С ними советская власть вела жестокую борьбу как в годы Великой Отечественной, так и после. Стали осуществляться переселения и депортации целых народов, которые поддерживали Гитлера.

В августе 1943 года советское правительство, опасаясь восстания, ликвидировало республику немцев Поволжья, а сами немцы были расселены по отдалённым регионам.

Депортации практиковались в северном Кавказе и в Крыму. Карачаевцев, балкарцев, чеченцев и ингушей обвинили в коллаборационизме. Такая же участь постигла калмыков, крымских татар, греков и ещё некоторые диаспоры, проживавшие здесь. Как правило, депортация сопровождалась ликвидацией национальных автономий, русификацией местной топономики.

\subsection{Военные события периода}

Начало 1944 года. Осуществляются стратегические операции советских войск в составе четырёх фронтов. Открыт выход на государственную границу в районе Карпат. Весной 1944 года очищен Крым.

Волховский, Прибалтийский и ещё какой-то фронты 27 января 1944 года окончательно сняли блокаду Ленинграда. Она продолжалась более 800 дней и унесла огромное количество жизней. Больше всего там погибло мирного населения. Официальное признанное число погибших в блокаде --- 630 тысяч человек, по современным оценкам --- более миллиона жертв. 90\% из них связаны со смертью от голода или заболеваний, связанных с ним.

6 июня 1944 года англо-американские силы под командованием генерала Эйзенхауера переправляются через Ла-Манш и высаживаются в Нормандии. На территории Нормандии англо-американские войска поддерживают формирования французского сопротивления --- силы, достигшей 500 тысяч бойцов.

В Париже началось восстание против оккупантов. К моменту подхода союзников город фактически принадлежал французам.

Также высадка состоялась в районе города Канны.

Для избежания окружения немцы выводят свои войска и останавливаются на оборонительной линиии Зигфельда.

Советские войска разворачивают мощное наступление в Карелии, Прибалтике, Белоруссии, Румынии.

Финляндия подписывает перемирие с СССР, а 4 марта 1944 года объявляет войну Германии. В ход операции по освобождению Белоруссии "Багратион" была разгромлена группа армий "Центр", освобождена Литва, Латвия и восток Польши.

Венгрия, Болгария, Югославия и Чехословакия при советской помощи освобождаются от нацизма. Фронт приблизился с запада и востока к границам Германии. Она оказалась фактически изолирована в тисках. Проводится тотальная мобилизация, в армию забирают подростков и стариков.

Силами десяти фронтов при поддержке Чехословацкого корпуса, Французского авиаполка советская армия наносит сокрушительное поражение силам противника.

Проводятся Восточно-Одерская и Висло-Прусская операции.

С 4 по 11 февраля 1945 года состоялась Ялтинская конференция. На ней согласованы действия по совершению разгрома Германии и послевоенному урегулированию. Советский союз брал на себя обязательства вступить войну с Японией максимум через 3 месяца после окончания войны в Европе.

Англо-французские войска к апрелю 1945 года выходят к Эльбе. К началу мая капитулировали остатки немецких сил в Голландии, Дании и на западе Германии.

Гитлер совершает самоубийство, власть получил адмирал Дениц, который надеялся заключить сепаратный мир с США и Великобритенией.

Советские войска заходят на территорию Германии, где 16 апреля -- 8 мая 1945 года проходила Берлинская операция, последняя операция Великой Отечественной Войны.

1-й Белорусский под командованием Жукова, 2-й Белорусский под командованием Рокоссовского и 1-й Украинский под командованием Конева при поддержке войска польского вели наступление на Берлин. Почти 500 тысяч немецких солдат попало в плен.

8 мая 1945 года немецкое командование подписывает Акт о безоговорочной капитуляции Германии.

Последняя крупная группировка врага была разгромлена в Праге 9 мая 1945 года. Она не имела связи со своим командованием или не хотела признавать капитуляцию, поэтому какое-то время продолжала сопротивление.

Состоялась Постдамская конференция, где главы Антигитлеровской коалиции приняли решение о послевоенном устройстве в Европе.

8 августа 1945 года Советский Союз объявляет войну Японии, захватившей Корею, значительную часть Китая и многие острова в Тихом океане.

Забайкальский, 1-й и 2-й Дальневосточный фронты при поддержке Тихоокеанского фронта под командованием маршала Василевского вторгаются на территорию Японии. У Квантунской армии было большое численное превосходство над Квантунской армии. При поддержке Монгольской народной республики в короткие сроки были освобождены Корея, Маньчжурия и Курильские острова.

Большую роль сыграла предшествовавшая борьба союзников против Японии, а также сброс ядерных бомб на Хиросиму и Нагасаки 6 и 9 августа соответственно. Применение ядерного оружия деморализовало японское правительство и общество.

2 сентября 1945 года японцы капитулировали.

Победа стран Антигитлеровской коалиции имела всемирное историческое значение. Она оказала влияние на судьбу человечества и продолжает оказывать до сих пор.

В ходе четырёхлетней борьбы гитлеровская Германия потеряла 80\% личного состава и 75\% техники на востоке. Цена Победы была огромная. По приблизительным подсчётам Советский Союз потерял по самым скромным оценкам 27 миллионов человек. Из них на фронте в боях погибло около 10 миллионов, половина из которых --- гражданское население, уничтоженное бомбёжками и в концентрационных лагерях. Около 6 млн советских граждан было угнано. 1710 городов, свыше 70 тысяч сёл и деревень Советского Союза уничтожено. Потрачено почти 2 миллиарда советских рублей, 30\% национального богатства утеряно. После войны осталась масса инвалидов, людей с осколочными ранениями и детей-сирот.

\section{1945-1953 гг. СССР в послевоенный период: экономика, социум, внешняя политика}

За время войны Советский Союз вошёл в число ведущих держав мира. Престиж страны заметно возрос. Во внутренней политике почти ничего не поменялось. Иосиф Виссарионович Сталин сохранял всю полноту власти.

В июне 1945 года Сталин за свои воинские заслуги получил звание генералиссимуса. За всю историю России Сталин стал пятым носителем этого титула.

Основная задача в послевоенные годы --- скорейшее восстановление послевоенного хозяйства. Нужно поставить экономику на мирные рельсы.

Началась постепенная демобилизация советской армии, составлявшей 11 миллионов человек. Бывшие солдаты отправляются на восстановление страны: на стройки, машинно-тракторные станции, в колхозы и совхозы. Проводится репатриация советских граждан. В СССР к концу 1945 года вернулись почти 5 миллионов человек, многие из которых стали спецпоселенцами: все возвращавшиеся проходили фильтрацию и проверялись на вербовку иностранными спецслужбами, на преступления в годы войны\dots В случае неблагоприятных результатов исследований такие граждане ограничивались в правах.

Восстанавливаются 8-часовой рабочий день, создаётся IV пятилетний план (1946-1950 гг.). Когда его составляли, исходили из того, что восстанавливать экономику нужно с тяжёлой промышленности --- металлургии, добычи полезных ископаемых. Перепрофилирование заводов, ранее переведённых на военные рельсы, также требовало времени.

Довоенный уровень промышленного производства был достигнут к 1948 году. Активно использовалось трофейное промышленное оборудование, а также полученное в рамках репараций.

Металлургическая и топливно-энергетическая база страны была быстро восстановлена. Возродились важнейшие отросли земледелия.

Восстановление сельского хозяйства в основном шло за счёт крестьянсва, обложенного большими налогами.

К 1950 году восстановился железнодорожный транспорт. Рос уровень жизни советских людей в городах.

В 1947 году была проведена денёжная реформа. Особенность каждой советской денёжной реформы заключается в том, что она наносила ущерб сбережения населения путём деноминации: старые рубли заменялись на новые в пропорции 10 к 1, но цены снижались непропорционально, а на обмен валюты давался короткий срок (неделья). Основные пострадавшие --- крестьяне. Чуть позже отменена карточная система продовольствия.

Временное увеличение благосостояния горожан связано с разорением деревни.

В 1946-1947 гг. случилась засуха в хлебопроизводящих регионах, что немедленно вызвало голод. В эти годы погибло от нескольких сотен тысяч до 1,5 млн человек. Отсутствие точных данных связано с тем, что переписи населения не проводились до 1959 года.

По плану Маршалла многие страны Европы получали помощь от США. Сталин отказывался от помощи запада. Государство приобретало продукцию сельского хозяйства по заниженной цене. На подсобные хозяйства государство всячески давило, в том числе повышенной ставкой налога.

В своей работе "Экономические проблемы социализма" Сталин писал о том, что советская экономика всегда будет дефицитной. Сталин выступил за постоянное развитие тяжёлой промышленности, а также утверждал, что растущие потребности населения всегда будут превышать возможности экономики.

\end{document}