\documentclass{book}

\usepackage[T2A]{fontenc}
\usepackage[utf8]{inputenc}
\usepackage[english,russian]{babel}
\usepackage{mathtools}
\title{Название}
\author{Даня Грозный}
%\with{Владимир Владимирович}

\begin{document}
Томпсон рассматривал круговые процессы. Его формулировка звучала

Основываясь на первом начале, получается, что оно не даёт информации о том, в какую сторону процесс протекает. В случае второго начала термодинамики мы можем определить, куда именно переходит тепло. Мы можем определить полезность работы и оценить тепло.

В имеющихся термодинамических системах у нас есть как обратимые, так и необратимые процессы. ОБратимые можно вернуть в исходное состояние, при необратимых кардинально изменяется термодинамическая характеристика. Например, в случае растягивания мы можем вернуть в исходное состояние только эластичные материалы. Резину можно, ткань --- нет.

Что касается цикличности. Это у нас два состояния замыкаются в единый цикл: у нас всегда есть прирост и убыль работы. Термодинамическая система обменивается теплотой как внутри себя, так и с внешней средой. Отсюда возникает ситуация, когда мы можем рассмотреть
\begin{itemize}
    \item {\bf Процесс тепловой машины.} У нас есть некий нагреватель, который будет давать определённое кол-во тепла для нашего рабочего тела. У нас есть возможность отвода энергии, т. е. у нас есть холодильник.

    Есть некий сосуд, поршень, объём. Положение поршня определяет первичное состояние. Рабочим телом будет являться объём газа, жидкости или аморфного твёрдого вещества. Если мы хотим передать рабочему телу тепло, то мы перемещаем наш поршень на новое расстояние. Мы изменяем объём и совершаем некоторую работу.

    В случае работы с отводом в холодильник мы должны вернуть в исходное положение тот объём, который у нас образовался.

    Чтобы работала тепловая машина, нам нужно совершать несколько циклов туда-сюда, чтобы количество тепла, создаваемое внутри рабочего тела, постоянно менялось.

    Основываясь на первом начале термодинамики, мы можем записать изменение термодинамических характеристик, которые происходят внутри нашего объёма.

    Исходя отсюда, мы можем сказать, что на наших графиках работа --- суммарная работа $A$ от всего происходящего процесса.

    В отношении работы может быть несколько ситуаций: когда они (обе работы) равны и когда они отличаются.

    Если работа совершается из состояния 1 в состояние 2, это у нас определяемая площадь по данной кривой.

    Во втором пути это та же самая площадь.

    Рассматривая три возможные ситуации, мы понимаем, что у нас для тепловой машины возможна только ситуация $A_{12} > A_{21}$.

    Почему мы не учитываем изменение внутренней энергии нашего рабочего тела? Несмотря на то, что у нас один и тот же газ и над ним совершается работа, внутренняя энергия в процесах уплотнения и расширения газа различна. Исходя из этого мы можем записать наше соотношение, то есть преддверие нашего закона, что у нас непосредственно \dots Чтобы получить полную работу, нам по сути нужно сложить две работы с учётом знаковости: первая работа минус вторая работа. Это будет полная работа, которая совершается данной термодинамической системой.

    С учётом того, что у нас тепловая машина, мы можем посчитать КПД. Обычно он определяется как отношение полной работы к выделяемому теплу. $\frac{A_1 - A_2}{Q}$ можно записать в виде разности $\frac{A_1}{Q} - \frac{A_2}{Q}$.

    Яркие примеры --- хлодильник и кондиционер. В случае с холодильником некоторые думают, что, открыв холодильник, они охладят комнату. Но это не поможет охладить комнату, потому что холодильная машина выделяет также тепло. Баланс, который мы пытаемся установить с забором тепла, постоянно восстанавливается.

    \item {\bf Цикл Карно} {\it Кто делал Cp / Cv, он наблюдал адиабатический процесс}

    Машина Карно прредставляет собой цилиндром с поршнем, но без клапанов. Мы можем говорить о том, что загоняемый в цилиндр газ, подчиняется закону идеального газа. Машина Карно --- машина идеального плана, повторить её в масштабах реальности невозможно.

    У нас совершается работа, причём работа получается из четырёх компонентов и наша задача состоит в том, чтобы выяснить эти самые компоненты.

    Рассмотрим изотермические процессы 1 и 2. Внутренняя энергия остаётся без изменения. Подводимое тепло будет полностью переходить в нашу работу. Из уравнения Менделеева -- Клайперона наша работа будет связана с изменением нашего объёма.

    В случае адиабатического процесса, то есть состояния 2, 3, мы будем смотреть приращение тепла, которое, как правило, равно нулю. Совершаемая работа будет отрицательная и зависит от изменения теплоёмкости при постоянном объёме.

    Возвращаемся в адиабатический процесс 3-4, где у нас будет отвод нашего тепла. У нас осталась последняя адиабата. Работа отсюда также будет по аналогии с предыдущей адиабаты со знаком минус.

    Нам осталось только собрать всё в единое целое. Чтобы найти работу и КПД нашего цикла, мы возьмём сумму всех наших работ. Всё замечательно складываем. Наше КПД будет определяться следующим соотношением \dots Получаем зависимость от изменения нашего объёма.

    Проводя дальше преобразование через уравнение Менделеева -- Клайперона, мы можем получить, что наше соотношение будет \dots

    В итоге мы получаем, что КПД цикла Карно определяется только начальной и конечной температурой. Отсюда вытикает две его теоремы:
    \begin{itemize}
        \item Теорема 1. КПД не зависит от вещества, которое мы используем. Достаточно знать только температуру.
        \item Теорема 2. КПД обратимой машины не может быть больше КПД необратимой.
    \end{itemize}
    \item {\bf Процесс холодильной машины.} В случае холодильной машины у нас ситуация меняется. Термодинамические параметры минимальны. Прирост тепла будет в обратную сторону. В тепловой машине мы теряем работу, в холодильной --- затрачиваем некое количество энергии на совершение этой самой работы. По аналогии с циклом Карно мы получим отношение температур.
    
    Если мы берём обычный наш холодильник, то КПД здесь равно 4,17. Чтобы охладить термодинамическую систему, нам нужно привести 4,17 Дж.
\end{itemize}

\section{Третье начало термодинамики (теорема Неркста)}
Энтропия при стремлении к нулю для незамкнутой системы будет определяться её предельным состоянием, не зависящим от давления, фазы и плотности.



29 и 30 мая инфиз не учится. Скорее всего, занятий не будет. Если занятие будет, то будут приниматься только отчёты, лабы провести не получится.
\end{document}