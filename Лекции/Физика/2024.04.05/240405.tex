\documentclass[12pt]{article}

\usepackage[T2A]{fontenc}
\usepackage{amsmath}
\usepackage{amssymb}

\begin{document}
\section{Слайд 1}
\par Диффузия переносит массу, внутреннее трение будет отвечать за импульс.
\section{Слайд 2}
\par Мы себя ограничиваем в некотором виде и говорим, что система изотропна по направлениям: по всем осям в нашем газе не будет встречаться нигде икаких однородностей.
\par У нас будут выполняться все законы сохранения массы, импульса и энергии.
\par Мы также допускаем, что неравновесный параметр нашей системы $\equiv const$.
\par Обязательное условие в случае модели идеального газа: мы будем считать, что все процессы у нас непрерывны.
\par Все элементы системы могут иметь произвольное значение... Даже после столкновения у нас будет определено значение скоростей и направлений движения молекул.
\section{Слайд 3}
Чтобы составить модель, мы представим, что у нас есть некое пространство. Мы будем пока рассматривать в одном направлении и размистим в одном отрезке площадку (контрольную, перпендикулярную направлению движения молекул), чтобы понять, сколько частиц будет проходить через эту площадку и количество столкновений в пределах этой площади.
\par Мы имеем перемещение двух веществ, считаем, что у нас два разных газа (либо разные скорости, либо разные размеры). Мы считаем, что концентрация, которая идёт слева направо, определяется как формула $N_+$. Мы взяли только одну возможность (только распространение в одном направлении), поэтому мы подставляем коэффициент $\frac 1 6$. $N_-$ -- то же самое для второго газа.
\par $\Delta N$ -- разность между столкновениями -- в случае одной скорости за $\Delta t$ мы считаем только разности концентраций. Если концентрации одинаковы, система однородная и находится в равновесии, потому что перемещение одних и тех же молекул с одинаковой скоростью не происходит.
\section{Слайд 4. Явление переноса}
\par Если концентрации различны, мы будем наблюдать несколько процессов и, в частности, диффузию.
\par Нам нужно найти коэффициент диффузии и, возможно, изменение.
\par У нас в примере на слайде газ, поэтому рассмотрим случай газообразных смесей. Чтобы получить коэффициент диффузии, мы обе части разделим и умножим на две длины свободного пробега $\vec \lambda$. Уможнив и разделив, мы ничего не поменяем, но при этом длина пробега будет определяться ($\Delta x - 2 \vec \lambda$)
\section{Слайд 5. Явление переноса}
Ых
\section{Слайд 6. Явление переноса}
Ых
\section{Слайд с фото 1. Явление переноса}
\par По принципу конвекции работают кондиционеры: холодный воздух опускается, горячий -- поднимается.
\par 
\section{Следующий слайд (с фото 2). Явление переноса}
\par Наше изменение (наш удельный тепловой поток) будет определяться непосредственно градиентом температуры $\frac{\Delta T}{\Delta x}$ и умножаем на удельную теплопроводность вещества, которая является табличным значением для каждого вещества.
\par Градиент определяет нижнюю и верхнюю границу температуры.
\section{Лекция 9. Элементы статистического метода исследования термодинамических систем}
\subsection{Слайд 1. Случайные величины и их описание. Фото 3}
\par Случайность даёт нам вероятность. Справебыдло.
\par Вероятность случайного события: $dN$ -- успехи события, $N$ -- все исходы.
\subsection{Слайд 2. Случайные величины и их описание. Фото 4}
\subsection{Слайд 3. Случайные величины и их описание. Фото 5}
\par При получении одного и того же вопроса по смысловой нагрузке с разной формулировкой студент будет давать ответ по-разному. Это literally флуктуация.
\par Параметры взомжного состояния ТДС. С помощью функции распределения мы можем сказать, будет ли в этой системе происходить то или другое, будут ли у неё предпосылки. 
\subsection{Фото 6. Случайные величины и их описание}
\par Непосредственный расчёт вероятности нас не интересует, у нас не математика.
\par Распределение Гиббса --- универсальное распределение: из него можно получить три остаольных.
\subsection{Фото 7. Следующий слайд. Распределение Максвелла}
\par У нас полный хаос. В нём лишь температура и ускорение силы тяжести $\equiv const$. Ускорение силы тяжести и ускорение свободного падения в данной ситуации --- близкие значения. Но вообще ускорение силы тяжести и ускорение свободного падения равны разве что в центре Земли.
\par График будет приближаться к OX, если T увеличивается
\subsection{Фото 8. Следующий слайд. Распределение Максвелла}
\par Наше распределение пропорционально квадтрату скорости
\par Увеличение температуры даёт смещение графика в более высокие скорости и подтверждает опыты: при увеличении скорости молекулы начинают двигаться быстрее.
\subsection{Фото 9. Следующий слайд. Распределение Максвелла}
\par Свойства распределения Максвелла
\par Относительно пункта 4. У нас нет строго определённой скорости, так как у нас градация по скоростям, поэтому значение конкретной скорости будет отсутствовать. Изменение скорости постоянно.
\subsection{Фото 10. Распределение Максвелла.}
\par С учётом того, что композиционные материалы в большинстве случаев -- твёрдые вещества, мы можем подбирать нужную кристаллическую структуру с помощью структуры Максвелла. И то же самое, например, с жидкостью.
\par Когда Максвелл своё распределение выводил, он его вывел и доказал математически. Экспериментальное подтверждение нашлось позже. В ходе вывода он ввёл ряд величин, которые теперь являются одними из основных при определении параметров термодинамической системы. Это средняя скорость (в отличие от механики, в молекулярной физике определяется только понятие средних скоростей -- количество возможных соударений на функцию распределения по скоростям и на саму скорость)
\subsection{Фото 11. Распределение Максвелла}
\par Вторая величина Максвелла -- средняя квадратичная скорость -- корень из квадрата скорости.
\par Среднее квадратичное отклонение. Если у меня какая-то величина $a_{\text{кв}} = \sqrt{\frac{a^2_{\text{ср}}}{n(n-1)}}$. Средняя квадратичная скорость связана с этим понятием
\subsection{Фото 12. Распределение Максвелла}
\subsection{Фото 13. Распределение Максвелла}
\par Экспериментальное подтверждение теории Максвелла.
\par Электроны сереряного покрытия. Мы начинаем вращать цилиндры с одинаковой угловой скоростью. Каждый раз у нас пролетает через щель.
\subsection{Фото через Kamoso. Распределение Максвелла}
\par Распределение молекул по импульсам и кинетическим энергиям.
\par Если мы засыпем песок в сферу, запаяем сферу, сделаем в ней отверстие и начнём её вращать вокруг своей оси. Высыпающийся песок будет повторять распределение Максвелла.
\subsection{Распределение Больцмана}
\par Распределение Больцмана -- это барометрическая формула
\par Мы находимся в модели идеального газа. Прироста и убывания вещества нет. Объём зафиксирован, он находится в поле силы тяжести (высота объёма будет играть значение).
\par Возьмём некоторый слой заданной площади $S$ и на этот слой на некоторой высоте $dh$. С точки зрения классического подхода мы будем иметь два процесса: воздействие силы тяжести и тепловое хаотическое движение молекул.
\par Для упрощения модели мы будем считать, что поле тяготения у нас одно. Мы находимся в одном месте пространства и $g \equiv const$.
\par Формула внизу -- это второй закон Ньютона, просто в непривычной записи.
\subsection{Фото 3. Распределение Больцмана}
\par Барометрическая формула
\par Буровые установки: измерение концентрации при движении вглубь земли.
\subsection{Фото 4. Распределение Больцмана}
\par Наше распределение будет определяться через потенциальную энергию.
\subsection{Фото 5. Распределение Больцмана}
\par Экспериментальное подтверждение -- Ольг Перрена



\end{document}