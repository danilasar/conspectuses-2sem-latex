\documentclass[12pt]{article}

\usepackage[T2A]{fontenc}
\usepackage{amsmath}
\usepackage{amssymb}

\title{Алгебра логических значений}
\author{Максим Андреевич}

\begin{document}
\par Пример алгебры даёт множество $\{0,1\}$ истинностных значений высказываний с $n$-арными операциями $F_\Phi$, которые являются функциями истинностных значений формул логики высказываний $\Phi=\Phi(X_1,\dots,X_n)$, образованных с помощью $n$ пропозициональных переменных $X_1, \dots, X_n$.
\par Формула $\Phi = \neg X$ определяет унарную операцию $F_\Phi = F_{\neg X}(x)$, которая обозначается символом $x'$ и называется \textit{отрицнием} или \textit{дополнением} переменной $X$.
\par Формулы $\Phi = X \lor Y$, $\Psi = X \land Y$ определяют бинарные операции $F_\Phi = F_{X \lor Y}(x, y)$, $F_\Psi = F_{X \land Y}(x, y)$, которые обозначаются соответственно символами $x \lor y$, $x \land y$ и называются \textit{дизъюнкцией} и \textit{конъюнкцией} переменных $x, y$.
\par Операция $x \lor y$ иногда также называется \textit{объединением} или \textit{суммой} переменных $x, y$ и обозначается соответственно через $x \cup y$ или $x + y$.
\par Операция $x \land y$ иногда также называется \textit{пересечением} или \textit{произведением} переменных $x, y$ и обозначается соответственно через $x \cap y$ или $x \cdot y$.
\par Историческкая справка. Алгебра $B=({0,1},\lor,\land,')$ впервые была введена в XIX веке английским математиком Дж. Булем с целью применения в логике математических методов.
\par Поэтому эта алгебра называется \textit{алгеброй Буля} или \textit{алгеброй логических значений}.
\par \underline{Теорема.} Алгебра Буля $B=({0,1},\lor,\land,')$ удовлетворяет свойствам:
\begin{enumerate}
    \item $a \lor (b \lor c) = (a \lor b) \lor c, a \land (b \land c) = (a \land b) \land c$ --- ассоциативность дизъюнкции и конъюнкции;
    \item $a \lor b = b \lor a$, $a \land b = b \land a$  --- коммутативность дизъюнкции и конъюнкции;
    \item $a \lor a = a$, $a \land a = a$ --- идемпотентность дизъюнкции и конъюнкции;
    \item $a \land (b \lor c) = (a \land b) \lor (a \land c)$, $a \lor (b \land c) = (a \lor b) \land (a \lor c)$ --- дистрибутивность соответственно конъюнкции относительно дизъюнкции и дизъюнкции относительно конъюнкции;
    \item $(a')' - a$ --- идемпотентность дополнения;
    \item $(a \lor b)' = a' \land b', (a \land b)' = a' \lor b'$ --- законы де Моргана;
    \item $a \lor (a \land b) = a, a \land (a \lor b) = a$ --- законы поглощения;
    \item $a \lor a' = 1$, $a \land a' = 0$ --- характеристическое свойство дополнения;
    \item $a \lor 1 = 1$, $a \land 1 = a$ --- характеристическое свойство наибольшего элемента 1;
    \item $a \lor 0 = a$, $a \lor 0 = 0$ --- характеристическое свойство наименьшего элемента 0.
\end{enumerate}
\par Для описания алгебраических свойств булевых алгебр используются Формулы, которые называются \textit{булевыми многочленами} и которые образованы из булевых переменных $x, y, \dots$ (принимающих значения 0, 1) и символов булевых операций $+, \cdot, '$ по следующим правилам:
\begin{enumerate}
    \item Все булевы переменные $x, y, \dots$ и символы 0, 1 --- булевы многочлены;
    \item Если $p$ и $q$ --- булевы многочлены, то таковыми являются выражения
    \begin{equation*}
        (p)+(q), (p) \cdot (q), (p)'
    \end{equation*}
\end{enumerate}
\par Если $p$ образован с помощью $x_1, \dots, x_n$, то он обозначается $p(x_1, \dots, x_n)$.
\par Множество всех булевых многочленов от $n$ переменных обозначим $P_n$.
\par Если в $p(x_1, \dots, x_n)$ вместо переменных $x_1, \dots, x_n$ подставить произвольные значения $a_1, \dots, a_n$ из множества $B$, то в результате вычислений получится некоторый элемент $\vec p(a_1, \dots, a_n)$ алгебры $B$.
\par Каждый булев многочлен $p(x_1, \dots, x_n)$ определяет отображение $\vec p: B^n \to B$, которое называется \textit{булевой полиномиальной функцией}, определяемой булевым многочленом $p(x_1, \dots, x_n)$.
\par \underline{Определение.} Булевы многочлены $p, q \in P_n$ называются \textit{эквивалентными}, если они определяют одну и ту же булеву полиномиальную функцию, т. е. $\vec p = \vec q$ ($\vec p ~ \vec q$, $\vec p \leftrightarrow \vec q$).
\par Бинарное отношение $~$ является эквивалентностью на множестве $P_n$.
\par Классы эквивалентности $[p]=\{q \in P_n : p ~ q\}$ образуют фактор-множество $P_n /~ = \{[p]: p \in P_n\}$.
\par Полные системы представителей этого фактор-множества назывюатся системами \textit{нормальных форм} булевых многочленов.
\par Для булевой переменной $x$ и $\alpha \in \{0,1\}$ положим:
\begin{equation}
    x^\alpha = \begin{cases}
        x, \text{ если } \alpha = 1,\\
        x', \text{ если } \alpha = 0.
    \end{cases}
\end{equation}
\par Выражение $x^\alpha$ называется \textit{литерой}.
\par Литера или конъюнкция (соответственно, дизъюнкция) литер называется \textit{конъюнктом} (соответственно, \textit{дизъюнктом}).
\par Конъюнкт (дизъюнкт) называется \textit{совершиенным}, если он содержит все булевы переменные рассматриваемой формулы.
\par Дизъюнкт или конъюнкция (совершенных) дизъюнктов называется (\textit{совершенной}) \textit{конъюнктивной нормальной формой}. Сокращённо КНФ и СКНФ, соответственно.
\par Конъюнкт или дизъюнкция (совершенных) конъюнктов называется (\textit{совершенной}) \textit{дизъюнктивной нормальной формой}. Сокращённо ДНФ и СДНФ, соответственно.
\par \underline{Теорема.} Любая булева функция $f:B^n \to B$ является булевой полиномиальной функцией следующих булевых многочленов:
\begin{equation*}
    p_f = \sum_{(\alpha_1,\dots,\alpha_n) \in B^n} f(\alpha_1, \dots, \alpha_n) \cdot x_1^{\alpha_1} \cdot \dots \cdot x_n^{\alpha_n}
\end{equation*}
\begin{equation*}
    q_f = \prod_{(\alpha_1,\dots,\alpha_n) \in B^n} (f(\alpha_1, \dots, \alpha_n) + x_1^{\alpha_1} + \dots + x_n^{\alpha_n})
\end{equation*}
\par \underline{Следствие 1}. Если булева функция $f:B^n \to B$ не равна тождественно нулю, то она является булевой полиномиальной функцией следующей СДНФ:
\begin{equation*}
    p_f = \sum_{(\alpha_1,\dots,\alpha_n) \in B^n, \\ f(\alpha_1, \dots, \alpha_n) = 1} x_1^{\alpha_1} \dots x_n^{\alpha_n},
\end{equation*}
которая называется \textit{СДНФ функции f}.
\par \underline{Следствие 2}. Если булева функция $f: B^n \to B$ не равна тождественно единице, то она является булевой полиномиальной функцией следующей СКНФ:
\begin{equation*}
    q_f = \prod_{(\alpha_1,\dots,\alpha_n) \in B^n, \\ f(\alpha_1, \dots, \alpha_n) = 0} (x_1^{\alpha_1} + \dots + x_n^{\alpha_n}),
\end{equation*}
которая называется \textit{СКНФ функции f}.

\end{document}