\documentclass[otchet]{SCWorks}
% Тип обучения (одно из значений):
%    bachelor   - бакалавриат (по умолчанию)
%    spec       - специальность
%    master     - магистратура
% Форма обучения (одно из значений):
%    och        - очное (по умолчанию)
%    zaoch      - заочное
% Тип работы (одно из значений):
%    coursework - курсовая работа (по умолчанию)
%    referat    - реферат
%  * otchet     - универсальный отчет
%  * nirjournal - журнал НИР
%  * digital    - итоговая работа для цифровой кафдры
%    diploma    - дипломная работа
%    pract      - отчет о научно-исследовательской работе
%    autoref    - автореферат выпускной работы
%    assignment - задание на выпускную квалификационную работу
%    review     - отзыв руководителя
%    critique   - рецензия на выпускную работу
% Включение шрифта
%    times      - включение шрифта Times New Roman (если установлен)
%                 по умолчанию выключен
\usepackage[T2A]{fontenc}
\usepackage[utf8]{inputenc}
\usepackage{graphicx}
\usepackage[sort,compress]{cite}
\usepackage{amsmath}
\usepackage{amssymb}
\usepackage{amsthm}
\usepackage{fancyvrb}
\usepackage{longtable}
\usepackage{array}
\usepackage[english,russian]{babel}
\usepackage{minted}
\usepackage{tempora}
\usepackage[hidelinks]{hyperref}

\usepackage{multirow}
\usepackage[table]{xcolor}\usepackage{longtable}\usepackage{array}
\usepackage{graphicx}%Вставка картинок правильная

\usepackage{float}%"Плавающие" картинки

\usepackage{cancel}%Зачёркивание

\usepackage{wrapfig}%Обтекание фигур (таблиц, картинок и прочего)
\setlength{\arrayrulewidth}{0.5mm}
\setlength{\tabcolsep}{18pt}

\newenvironment{centeritemize*}[1][]
  {\par\centering\begin{itemize*}[itemjoin=\quad,#1]}
  {\end{itemize*}\par}
\newcolumntype{C}[1]{>{\centering\let\newline\\\arraybackslash\hspace{0pt}}m{#1}}


\begin{document}

% Кафедра (в родительном падеже)
\chair{математической кибернетики и компьютерных наук}

% Тема работы
\title{Определение скорости звука в воздухе методом интерференции}

% Курс
\course{1}

% Группа
\group{151}

% Факультет (в родительном падеже) (по умолчанию "факультета КНиИТ")
% \department{факультета КНиИТ}

% Специальность/направление код - наименование
% \napravlenie{02.03.02 "--- Фундаментальная информатика и информационные технологии}
% \napravlenie{02.03.01 "--- Математическое обеспечение и администрирование информационных систем}
% \napravlenie{09.03.01 "--- Информатика и вычислительная техника}
\napravlenie{09.03.04 "--- Программная инженерия}
% \napravlenie{10.05.01 "--- Компьютерная безопасность}

% Для студентки. Для работы студента следующая команда не нужна.
\studenttitle{Студентов}

% Фамилия, имя, отчество в родительном падеже
\author{Григорьева Данилы Евгеньевича и Лазаревой Виктории Владимировны}

% Заведующий кафедрой 
\chtitle{доцент, к.\,ф.-м.\,н.}
\chname{С.\,В.\,Миронов}

% Руководитель ДПП ПП для цифровой кафедры (перекрывает заведующего кафедры)
% \chpretitle{
%     заведующий кафедрой математических основ информатики и олимпиадного\\
%     программирования на базе МАОУ <<Ф"=Т лицей №1>>
% }
% \chtitle{г. Саратов, к.\,ф.-м.\,н., доцент}
% \chname{Кондратова\, Ю.\,Н.}

% Научный руководитель (для реферата преподаватель проверяющий работу)
\satitle{доцент, к.\,ф.-м.\,н.} %должность, степень, звание
\saname{О.\,А.\,Черкассова}

% Руководитель практики от организации (руководитель для цифровой кафедры)
\patitle{доцент, к.\,ф.-м.\,н.}
\paname{С.\,В.\,Миронов}

% Руководитель НИР
\nirtitle{доцент, к.\,п.\,н.} % степень, звание
\nirname{В.\,А.\,Векслер}

% Семестр (только для практики, для остальных типов работ не используется)
\term{2}

% Наименование практики (только для практики, для остальных типов работ не
% используется)
\practtype{учебная}

% Продолжительность практики (количество недель) (только для практики, для
% остальных типов работ не используется)
\duration{2}

% Даты начала и окончания практики (только для практики, для остальных типов
% работ не используется)
\practStart{01.07.2022}
\practFinish{13.01.2023}

% Год выполнения отчета
\date{2023}

\maketitle

% Включение нумерации рисунков, формул и таблиц по разделам (по умолчанию -
% нумерация сквозная) (допускается оба вида нумерации)
\secNumbering

\tableofcontents

% Раздел "Обозначения и сокращения". Может отсутствовать в работе
% \abbreviations
% \begin{description}
%     \item ... "--- ...
%     \item ... "--- ...
% \end{description}

% Раздел "Определения". Может отсутствовать в работе
% \definitions

% Раздел "Определения, обозначения и сокращения". Может отсутствовать в работе.
% Если присутствует, то заменяет собой разделы "Обозначения и сокращения" и
% "Определения"
% \defabbr

\newpage

\section{Логика предикатов}
кр будет проводиться во время лекции, предварительно 29 марта.
\par \underline{Определение.} {\it Предикатом} называется утвержение, содержащее переменные $x_1,\dots,x_n$, которое превращается в высказывание при замене этих переменныз конкретными объектами из некоторой области возможных значений.
\par Обозначаются предикаты $P$, $Q$, $\dots$.
\par Переменные $x_1,\dots,x_n$ называются {\it предметными} или {\it индивидуальными переменными}. Число предметных переменных в предикате называется его {\it арностью} или {\it местностью}.
\par Более точно, предикат $P$ с $n$ предметныи переменными называется {\it $n$-арным} или {\it $n$-местным предикатом} и обозначается $P(x_1,\dots,x_n)$.
\par $M^n = \{(a_n, a_m): a_1, a_n \in M \}$
\par \underline{Определение.} {\it Предикатом} называется утверждение, содержащее переменные $x_1, \dots, x_n$, которое превращается в высказывание при замене этих переменных конкретными объектами из некоторой области возможных значений $M$.
\par Истинностная функция предиката...
\par \underline{Определение.} Предикат $P(x_1, \dots, x_n)$ на множестве M называется:
\begin{itemize}
    \item {\it тождественно истинным}, если $\forall x_1 = a_1 \in M, \dots, x_n = a_n \in M$ высказывание $P(x_1, \dots, x_n)$ истинно, т. е. $P^+ = M^n$
    \item {\it тождественно ложным}, если $\forall x_1 = a_1 \in M, \dots, x_n = a_n \in M$ высказывание $P(x_1, \dots, x_n)$ истинно, т. е. $P^+ = \varnothing$
    \item {\it выполнимым}, если $\exists x_1 = a_1 \in M, \dots, x_n = a_n \in M$ высказывание $P(x_1, \dots, x_n)$ истинно, т. е. $P^+ \neq \varnothing$
    \item {\it опровержимым}, если $\exists x_1 = a_1 \in M, \dots, x_n = a_n \in M$ высказывание $P(x_1, \dots, x_n)$ истинно, т. е. $P^+ \neq M^n$
\end{itemize}

\section{Алгебра предикатов}
\par {\it Отрицание $n$-местного предиката $P(x_1, \dots, x_n)$} определяетя как $n$-местный предикат $\neg P$, который при подстановке значений превращается в высказывание $\neg P(a_1, \dots, a_n)$, являющееся отрицанием высказывания $P(a_1, \dots, a_n)$.
\par {\it Конъюнокция $n$-местных предикатов $P(x_1, \dots, x_n)$ и $Q(x_1, \dots, x_n)$} определяется как $n$-местный предикат $P \land Q$, который при подстановке значений превращается в высказывание $P\land Q(a_1, \dots, a_n)$.
\par Для любого множества M допустимых значений переменных предикатов множества истинности предикатов взаимосвязаны с логическими операциями по следующим формулам: \\
$(\neg P)^+ =$ $M^n \textbackslash P^+$ \\
$(P \land Q)^+ =$ $P^+ \cap Q^+$ \\
$(P \lor Q)^+ =$ $P^+ \cup Q^+$ \\
$(P \Rightarrow Q)^+ =$ $(\neg P)^+ \cup Q^+$ \\
$(P \Leftrightarrow Q)^+ =$ $(P \Rightarrow Q)^+ \cap (Q \Rightarrow P)^+ =$ $(P^+ \cap Q^+) \cup ((\neg P)^+ \cap (\not Q)^+)$
\par {\it Примеры.}
\begin{enumerate}
    \item Пусть на множестве вещественных чисел $\mathbb R$ предикат $P(x)$ выражается неравенством $f(x) \leq 0$ и предикат $Q(x)$ выражается неравенством $g(x) \leq 0$. Тогда система неравенств $\begin{cases} f(x) \leq 0, \\ g(x) \leq 0 \end{cases}$ определяется как конъюнкция предикатов $P \land Q$ $\Rightarrow$ имеет множество решений $(P \land Q)^+ = P^+ \cap Q^+$, равное пересечению множеств решений неравенств системы.
    \item Пусть на множестве вещественных чисел $\mathbb R$ предикат $P(x)$ выражается неравенством $f(x) \leq 0$ и предикат $Q(x)$ выражается неравенством $g(x) \leq 0$. Тогда совокупность неравенств $\left[ \begin{gathered} f(x) \leq 0, \\ g(x) \leq 0 \end{gathered} \right.$ определяется как дизъюнкция предикатов $P \lor Q$ $\Rightarrow$ имеет множество решений $(P \lor Q)^+ = P^+ \cup Q^+$, равное объединению множеств решений неравенств системы.
\end{enumerate}

\par $\forall$ -- квантор общности (читается "для всех" - от All), $\exists$ -- квантор существования (читается "существует" - от Exist)
\par {\it Определение.} Результатом действия квантора общности $(\forall x_1)$ по переменной $x_1$ на $n$-местный предикат $P(x_1,\dots, x_n)$ называется $(n-1)$-местный предикат $(\forall x_1)P(x_1,x_2,\dots,x_n)$, который зависит от переменных $x_2,\dots,x_n$ и который при значениях $x_2=a_2,\dots,x_n=a_n$ в том и только том случае истинен на множестве $M$ допустимых значений переменной $x_1$, если при любых значениях $x_1 = a_1 \in M$ высказывание $P(a_1, a_2, \dots, a_n)$ истинно. \\
$(\forall x_1)P(x_1,x_2,\dots,x_n)$ $\overset{df}\Leftrightarrow$ при любых значениях $x_1 = a_1 \in M$ высказывание $P(a_1, a_2, \dots, a_n)$ истинно. \\
$(\forall x_1)P(x_1,x_2,\dots,x_n)$ -- предикат от переменных $x_2,\dots,x_n$ \\
при $x_2 = a_2,\dots,x_n = a_n$ истеннен на $M$ $\Leftrightarrow$ предикат $P(x_1, a_2, \dots, a_n)$ тождественно истинен на M.

\par {\it Определение.} Результатом действия квантора существования $(\exists x_1)$ по переменной $x_1$ на $n$-местный предикат $P(x_1,\dots, x_n)$ называется $(n-1)$-местный предикат $(\exists x_1)P(x_1,x_2,\dots,x_n)$, который зависит от переменных $x_2,\dots,x_n$ и который при значениях $x_2=a_2,\dots,x_n=a_n$ в том и только том случае истинен на множестве $M$ допустимых значений переменной $x_1$, если при некотором значении $x_1 = a_1 \in M$ высказывание $P(a_1, a_2, \dots, a_n)$ истинно. \\
$(\exists x_1)P(x_1,x_2,\dots,x_n)$ $\overset{df}\Leftrightarrow$ при хотя бы одном значении $x_1 = a_1 \in M$ высказывание $P(a_1, a_2, \dots, a_n)$ истинно.

\par Другие кванторы, как правило, являются сокращениями формул.

\par {\it Определение. Квантор существования и единственности} $(\exists ! x)P(x) = (\exists x)(P(x) \land ((\forall y)(P(y) \Rightarrow x = y)))$

\par Пример. \\
$\underset{P_1(\epsilon)}{\underbrace{(\forall \epsilon > 0)}} \underset{P_2(\delta)}{\underbrace{(\exists \delta > 0)}}$ \\
$(\exists Q(x))P(x) \overset{df}= (\exists x)(Q(x) \land P(x))$
$(\forall Q(x))P(x) \overset{df}= (\forall x)(Q(x) \underset{\bcancel{\cancel{\land}}}\Rightarrow P(x))$

\section{Алгебра предикатов}
Дописать определение... todo
\subsection{Формулы алгебры предикатов}
\par Свойства алгебры предикатов $P$ описываются с помощью специальных формул, которые строятся из символов предикатов и предметных переменных с помощью специальных вспомогательных символов -- скобок и знаков логических операций.
\par {\it Алфавит} алгебры предикатов состоит из следующих символов:
\begin{enumerate}
    \item {\it предметные переменные $x_1,x_2,\dots$}, которые используются для обозначения элеметнов множества допустимых значений
    \item $n$-местные {\it предикатные символы $P,Q,\dots$}, которые используются для обозначения $n$-местных предикатов на множестве допустимых значений
    \item символы логических операций $\neg, \land, \lor, \Rightarrow, \Leftrightarrow, \forall, \exists$
    \item вспомогательные символы (скобки, запятая и другие)
\end{enumerate}
{\renewcommand{\arraystretch}{1.5}
\setlength{\tabcolsep}{5pt}
\rowcolors{3}{black!10!white!50}{black!2!white!90}
    \begin{longtable}[h!]{ |c|c| }
        \hline
        Вспом. выс. & Лог. пред \\
        \hline
        \endhead
        $X$ & $P(x_1, \dots, x_n)$ \\
        \hline
    \end{longtable}}

\par {\it Формулы} алгебры предикатов определяются по индукции следующим образом:
\begin{enumerate}
    \item для любого $n$-местного предикатного символа $P$ и любых $n$ предметных переменных $x_1, \dots, x_n$ выражение $P(x_1, \dots, x_n)$ есть формула, которая называется {\it элементарной} (или {\it атомарной}) {\it формулой};
    \item если $\Phi, \Psi$ -- формулы, то формулами являются также выражения: $(\neg \Phi), (\Phi \land \Psi), (\Phi \lor \Psi), (\Phi \Rightarrow \Psi), (\Phi \Leftrightarrow \Psi)$
    \item если $\Phi$ - формула и $x$ - предметная переменная, то формулами являются также выражения $(\forall x)\Phi$, $(\exists x)\Phi$; при этом переменная $x$ и формула $\Phi$ называется {\it областью действия} соответствующего квантора.
\end{enumerate}
Приоритеты: кванторы, отрицание, конъюнкция, дизъюнкция и остальные. \\
{\it Пример. } $\underset{3}{\underline{\underset{2}{\underline{\underset{1}{\underline{(\forall x)}}P(x)}} \underset{\text{\it не зависит от } \forall}{\land Q(x)}}}$
\par Если в формулу $\Phi$ входят переменные $x_1, \dots, x_n$, то записывают $\Phi = \Phi(x_1, \dots, x_n)$.
\par Вхождение предметной переенной $x$ в формулу $\Phi$ называется {\it связным}, если она находится в области действия одного из кванторов по этой переменной. В противном случае вхождение называется {\it свободным}.
\par замкнут. и незамкнут. todo
\subsection{Интерпретации формул алгебры предикатов}
\par {\it Область интерпретации} -- непустое множество $M$, которое является областью возможных значений всех предметных переменных.
\par $n$-местным предикатным символам $P$ присваиваются конкретные значения $P_M$ $n$-местных предикатов на множестве $M$.
\par Соответствие $\beta: P |\Rightarrow P_M$ называется {\it интерпретацией предикатных символов}.
\par Область интерпретации $M$ вместе с интерпретацией предикатных символов $\beta$ называется {\it интерпретацией формул алгебры предикатов} и обозначается $(M, \beta)$ или просто $M$. \\
{\it Пример. }$M \neq \varnothing$ \\
\par При наличии интерпретации $M$ конкретные значения предметным переменным формул алгебры предикатов присваиваются с помощью отображения $\alpha$ множества всех предметных переменных $X$ d j,kfcnm bynthghtnfwbb $M$. Такие интерпретации называются {\it оценками}
\par {\it Выполнимость формулы} $\Phi$ в интерпретации $M$ при оценке $\alpha$ обозначается $M |=_\alpha \Phi$ -- читается "формула $\Phi$ истинна в интерпретации M при оценке $\alpha$" и определяется следующим образом:
\begin{enumerate}
    \item если $\Phi = P(x_1, \dots, x_n)$ для $n$-местного предикатного символа $P$ и предметных переменных $x_1, \dots, x_n$, то $M |=_\alpha \Phi$ тогда и только тогда, когда высказывание $P_M(\alpha(x_1),\dots,\alpha(x_n))$ истинно;
    \item если $\Phi = \neg \Psi$ для формулы $\Psi$, то $M |=_\alpha \Phi$ $\Leftrightarrow$ неверно, что $M |=_\alpha \Psi$;
    \item если $\Phi = \Phi_1 \square \Phi_2$ для формул $\Phi_1, \Phi_2$, то $M |=_\alpha \Phi$ $\Leftrightarrow$ $M |=_\alpha \Phi_1$ и $M |=_\alpha \Phi_2$
    \item если $\Phi = \Phi_1 \lor \Phi_2$ для формул $\Phi_1, \Phi_2$, то $M |=_\alpha \Phi$ $\Leftrightarrow$ $M |=_\alpha \Phi_1$ или $M |=_\alpha \Phi_2$
    \item если $\Phi = \Phi_1 \lor \Phi_2$ для формул $\Phi_1, \Phi_2$, то $M |=_\alpha \Phi$ $\Leftrightarrow$ неверно, что $M |=_\alpha \Phi_1$ и $M |=_\alpha \neg \Phi_2$
\end{enumerate}


% \conclusion

% Библиографический список, составленный вручную, без использования BibTeX
%
% \begin{thebibliography}{99}
%   \bibitem{Ione} Источник 1.
%   \bibitem{Itwo} Источник 2
% \end{thebibliography}

% Отобразить все источники. Даже те, на которые нет ссылок.
% \nocite{*}

% Меняем inputencoding на лету, чтобы работать с библиографией в кодировке
% `cp1251', в то время как остальной документ находится в кодировке `utf8'
% Credit: Никита Рыданов
\inputencoding{cp1251}
% \bibliographystyle{gost780uv}
% \bibliography{thesis}
\inputencoding{utf8}

% При использовании biblatex вместо bibtex
% \printbibliography

% Окончание основного документа и начало приложений Каждая последующая секция
% документа будет являться приложением
\appendix

\end{document}