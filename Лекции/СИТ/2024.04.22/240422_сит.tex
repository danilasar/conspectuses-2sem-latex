\documentclass{article}

\usepackage[T2A]{fontenc}
\usepackage[utf8]{inputenc}
\usepackage[english,russian]{babel}
\usepackage{mathtools}
\usepackage{listings}
\usepackage{xcolor}
\usepackage{qrcode}
\usepackage{outlines}
\usepackage{minted}

\usepackage{float} % ТРАХАТЬ ЛАТЕХ В ЖОПУ

\setminted[rust]{linenos,breaklines=true}

\begin{document}
\begin{figure}[H] % h, h!, H - здесь; t - top; b - bottom
    \centering
    \label{figure:zasadil}
    \caption{Засадил}
    \includegraphics[width=250px]{surovikin.jpg}
\end{figure}
Пакет minted, когда мы его объявляем и пишем код, делает запрос к shell. По умолчанию так делать нельзя, поэтому есть флаг -\--shell-escape. На питон нужно установить пакет pygments. 
\begin{minted}{rust}
    use std::io;

// n - номер числа Фибоначчи
fn fib(n: u8) -> Result<(), String> {
    let dp:&mut [u128; 187] = &mut [0u128;187];
    if n == 0 {
        return Err(String::from("Ноль не является номером"));
    }
    if n > 187 {
        return Err(String::from("Программа способна вычислить только первые 187 чисел ряда"));
    }
    dp[0] = 0;
    println!("{}", dp[0]);
    dp[1] = 1;
    if n >= 2 {
        println!("{}", dp[1]);
    }
    let mut i:usize = 2usize;
    while i < usize::from(n) {
        dp[i] = dp[i - 1] + dp[i - 2];
        println!("{}", dp[i]);
        i += 1;
    }
    return Ok(());
}

fn main() {
    println!("Введите номер желаемого числа из ряда Фибоначчи: ");
    let mut input = String::new();
    io::stdin()
        .read_line(&mut input)
        .expect("Не могу читать входящий поток");
    let trimmed = input.trim(); // избавляемся от пробелов и прочей лабуды
    match trimmed.parse::<u8>() {
        Ok(n) => {
            fib(n).unwrap_or_else(|error| {
                println!("{error}");
            });
        },
        Err(..) => println!("Это не число: {}", trimmed)
    };
}
\end{minted}
Вся информация на overleaf minted.

Кроме minted есть listings, vevlo, verbatim.

Код не должен занимать больше половины страницы. Файлы с ресурсами хранить в подкаталогах. Каждую секцию делать в отдельном .tex-файле, соответственно файлы секции тоже хранить в отдельном подкаталоге sections.

Преамбулу стоит выносить в файл preamble.sty. Это всё импортируется с помощью $\backslash $input\{preamble.sty\}.
\end{document}