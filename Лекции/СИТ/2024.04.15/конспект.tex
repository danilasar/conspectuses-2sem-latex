\documentclass{book}

\usepackage[T2A]{fontenc}
\usepackage[utf8]{inputenc}
\usepackage[english,russian]{babel}
\usepackage{mathtools}
\title{Название}
\author{Даня Грозный}
%\with{Владимир Владимирович}

\begin{document}
    \chapter{\LaTeX\LaTeX\LaTeX\LaTeX\LaTeX\LaTeX}
    \section{Секция}
    \subsection{Подсекция}
    \subsubsection{Подподсекция}
    \par На курсе есть лекция 1 с краткой сводкой предыдущего занятия
    \section{Чего не было на прошлом занятии?}

    Каким образом работает расположение текста в пределах документа?

    Какое"=то предложение. Другое предложение.

    Оставив пустой 19 строчку, мы выделяем новый абзац.

    Если мы используем чрезмерно длинное слово (длиннее строки) либо используется \textbackslash verb, у нас будут проблемы.

    Инлайн"=формула $i_{z + 3}n^{y^z}t^3e^{25}_2am$

    Выносная формула
    \begin{equation}
        \label{eq:sex}
        i_{z + 3}n^{y^z}t^3e^{25}_2am
    \end{equation}

    Вариант формулы \ref{eq:sex} без нумерации из asmmath:
    \begin{equation*}
        d_{z + 3}o^{y^z}gs^3\epsilon^{25}_2x
    \end{equation*}

    Греческий алфавит в формулах:
    \begin{equation}
        \beta^\rho\Delta^\sigma s^rM^\eta \leq  \geq
    \end{equation}

    \begin{equation}
        g(x) = \sum_{i = 0}^n a_i y_i
    \end{equation}

    \begin{equation}
        c(x) = \prod_{i = 0}^n h_i l_i d
    \end{equation}

    \begin{equation}
        f(x) = \int_{r = 0}^n u_r \cdot r_y
    \end{equation}

    \begin{equation}
        f(x) = \int\limits_{i = 0}^n u_i c_i k
    \end{equation}

    \begin{equation}
        \max{\frac{\sqrt[e]{\cos \sin \pi}}{x_0 + x_1 + x_ 2}}
    \end{equation}

   \begin{equation}
    a =
        \begin{matrix}
            1 & 2 & 3 \\
            4 & 5 & 6 \\
            7 & 8 & 9
        \end{matrix}
    \end{equation}

    \begin{equation}
        a =
            \begin{vmatrix}
                1 & 2 & 3 \\
                4 & 5 & 6 \\
                7 & 8 & 9
            \end{vmatrix}
    \end{equation}

    \begin{equation}
        a =
            \begin{bmatrix}
                1 & 2 & 3 \\
                4 & 5 & 6 \\
                7 & 8 & 9
            \end{bmatrix}
    \end{equation}

    \begin{equation}
        a =
            \begin{pmatrix}
                1 & 2 & 3 \\
                4 & 5 & 6 \\
                7 & 8 & 9
            \end{pmatrix}
    \end{equation}

    \begin{equation}
        a =
            \begin{Vmatrix}
                1 & 2 & 3 \\
                4 & 5 & 6 \\
                7 & 8 & 9
            \end{Vmatrix}
    \end{equation}

    \begin{equation}
        a =
            \begin{cases}
                1, ~ \text{если} ~ x > 0 \\
                0, ~ \text{иначе}
            \end{cases}
    \end{equation}

    \begin{equation}
        y(x) = \lceil 3 \left(\frac{\frac{2}{3}}{5}\right) \rceil
    \end{equation}


    \LaTeX\LaTeX\LaTeX\LaTeX\LaTeX\LaTeX\LaTeX\LaTeX\LaTeX\LaTeX\LaTeX\LaTeX\LaTeX

    Требования к работе:
    \begin{enumerate}
        \item Раскидываем по секциям
        \item Показываем знание раличного функционала \LaTeX
    \end{enumerate}

\end{document}