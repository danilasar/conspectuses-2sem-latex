\documentclass{book}

\usepackage[T2A]{fontenc}
\usepackage[utf8]{inputenc}
\usepackage[english,russian]{babel}
\usepackage{mathtools}
\usepackage{listings}
\usepackage{minted}
\usepackage{xcolor}
\usepackage{qrcode}
\usepackage{outlines}


\definecolor{codegreen}{rgb}{0,0.6,0}
\definecolor{codegray}{rgb}{0.5,0.5,0.5}
\definecolor{codepurple}{rgb}{0.58,0,0.82}
%\definecolor{backcolour}{rgb}{0.95,0.95,0.92}

\lstdefinestyle{lststyle:golang}{
    %backgroundcolor=\color{backcolour},   
    commentstyle=\color{codegreen},
    keywordstyle=\color{magenta},
    numberstyle=\tiny\color{codegray},
    stringstyle=\color{codepurple},
    basicstyle=\ttfamily\footnotesize,
    breakatwhitespace=false,         
    breaklines=true,                 
    captionpos=b,                    
    keepspaces=true,                 
    numbers=left,                    
    numbersep=5pt,                  
    showspaces=false,                
    showstringspaces=false,
    showtabs=false,                  
    tabsize=2
}

\lstdefinestyle{lststyle:sh}{
    %backgroundcolor=\color{backcolour},   
    commentstyle=\color{codegreen},
    keywordstyle=\color{magenta},
    numberstyle=\tiny\color{codegray},
    stringstyle=\color{codepurple},
    basicstyle=\ttfamily\footnotesize,
    breakatwhitespace=false,         
    breaklines=true,                 
    captionpos=b,                    
    keepspaces=true,                 
    numbers=left,                    
    numbersep=5pt,                  
    showspaces=false,                
    showstringspaces=false,
    showtabs=false,                  
    tabsize=2
}

\title{Курс молодого бойца: Golang}
\author{Даня Грозный}
%\with{Владимир Владимирович}

\begin{document}
\chapter{Основы}

%\begin{lstlisting}[language=go, caption=Пример работы strings.Builder, style=lststyle:golang]
\begin{minted}[linenos]{go}
    package main

    import (
        "fmt"
        "strings"
    )
    
    func main() {
        var str strings.Builder
    
        str.WriteString("a")
        str.WriteString("b")
        str.WriteString("c")
    
        fmt.Println(str.String())
    }
\end{minted}
%\end{lstlisting}
\section{Утилита Go}
Базовая программа, с которой начинается работа на языке Golang, -- $go$. Этот инструмент используется:
\begin{itemize}
    \item Для обновления зависимостей
    \item Для сборки и тестирования проектами
    \item Для управления артефактами
    \item Для форматирования исходного кода
\end{itemize}

Всё что нужно для работы на Go содержит в себе эта утилита. Отдалённо её можно сравнить с \textit{gcc}, хотя в сравнении с \textit{gcc} в \textit{go} вложен гораздо больший функционал. С полным её функционалом можно ознакомиться на официальном сайте:
\begin{center}
    \begin{tabular}{cc}
        https://pkg.go.dev/cmd/go & \qrcode{https://pkg.go.dev/cmd/go}
    \end{tabular}
\end{center}

Ежедневно Go-разработчик сталкивается с небольшим набором самых необходимых команд. Среди них:
\begin{outline}
    \1 $go~build$ компилирует проект и создаёт исполняемый бинарный файл
    \2 $go~build-race$ проверяет наличие проблем с параллелизмом/ конкурентностью. Если вы написали программу с использованием многопоточности, что довольно распространено, вам необходимо проверить свой код, чтобы выявить любые ошибки.
    \1 $go~run$ запускает проект без создания исполняемого файла (то есть работает в режиме интерпретатора). Это полезно при тестировании кода.
    \1 $go~mod$ управляет модулями и зависимостями, а именно файлом go.mod
    \2 $go~mod~tidy$ наводит порядок в файле \textit{go.mod}, загружает и обновляет зависимости
    \1 $go~test$ запускает тестовый набор проектами. Сюда полезно писать тесты для своего кода, чтобы убедиться, что он сохраняет правильную логику после обновления.
    \1 $go~fmt$ форматирует все файлы исходного кода согласно рекомендациям от Google
\end{outline}

\section{Аргументы командной строки}
Как известно, многие программы, в особенности утилиты командной строки, для параметеризации выполнения используют аргументы командной строки. На примере утилиты \textit{apt"=get}, использующейся для работы с пакетами в ряде дистрибутивов ОС Linux (в частности, Alt Linux и всего семейства Debian):
\begin{lstlisting}[language=sh]
    apt-get install go
\end{lstlisting}
Здесь:
\begin{itemize}
    \item \textit{apt"=get} --- программа, которую мы хотим выполнить
    \item $install$ и $go$ --- аргументы командной строки. Получив и интерпретировав их, \textit{apt"=get} поймёт, что от неё требуется найти и установить пакет $go$
\end{itemize}

Это же мы видели на примере уже рассмотренной утилиты \textit{go}: \textit{go run hello.go} передаст программе \textit{go} аргументы \textit{run} и \textit{hello.go}.

Возможность работать с аргументами командной строки предусмотрена и в языке Go.

Массив строк os.Args предоставляет собой необработанные аргументы командной строки. Обратите внимание, что первое значение в этом срезе - это путь к программе, а os.Args [1:] содержит аргументы программы. Напишем код, работающий с элементами этого массива:

\lstinputlisting[language=go,style=lststyle:golang,caption={Скрипт, выводящий аргументы командной строки на экран}]{010-os.args.go}

Чтобы поэкспериментировать с аргументами командной строки, лучше сначала создать двоичный файл с помощью go build. Поэтому выполним в терминале:

\lstinputlisting[language=sh,style=lststyle:sh]{011-os.args.sh}
На вывод мы получим следующий текст:
\begin{lstlisting}[style=lststyle:sh]
[./command-line-arguments a b c d]       
[a b c d]
c
\end{lstlisting}

\section{Переменные}
\subsection{Что такое переменные?}

Переменные представляют способ для доступа к данным в оперативной памяти, которые могут быть любыми, то есть изменчивыми.

По сути, переменная --- это просто псевдоним некоторой области памяти компьютера. Она нужна для упрощения доступа к данным, хранящимся там.

Чтобы сохранить информацию в области памяти, на которую ссылается переменная, нужно выполнить операцию присваивания (в языке Go её принято обозначать как \textit{:=}).

Переменная объединяет в себе несколько понятий:
\begin{itemize}
    \item Идентификатор --- имя переменной, которым мы оперируем в коде
    \item Данные, которые хранятся в выделенной области памяти (факт отсутствия данных --- тоже данные)
    \item Тип данных (как интерпретировать двоичный код, хранящийся в памяти: это может быть числом, текстом, структурой или чем-либо ещё)
\end{itemize}

\subsection{Объявление и инициализация переменной}

Создать (то есть объявить и инциализировать) переменную можно тремя способами. Рассмотрим их на примере \textit{example}, в которую будет записано число 3.
\begin{itemize}
    \item Мы объявляем переменную \textit{example} с помощью ключевого слова \textit{var} и тут же инициализируем её, присваивая ей значение 3. Go автоматически определит её тип данных --- целое число (\textit{int}).
\begin{lstlisting}[language=go,style=lststyle:golang]
var example = 3
\end{lstlisting}

    \item Другой вариант аналогичен предыдущему, но здесь мы явно указываем тип данных переменной.
\begin{lstlisting}[language=go,style=lststyle:golang]
var example int = 3
\end{lstlisting}

    \item И, наконец, мы можем сначала объявить переменную, а позже и инициализировать её, присвоив значение 3.
\begin{lstlisting}[language=go,style=lststyle:golang]
var example int
example = 3
\end{lstlisting}

\end{itemize}

\subsection{Комплексное создание переменных. Блоки}
Мы можем создать и инициализировать несколько переменных одной строчкой, перечислив их имена и значения следующим образом:
\begin{lstlisting}[language=go,style=lststyle:golang]
var a, b, c = 1, 2, "sample"
\end{lstlisting}

Тогда \textit{a} будет иметь значение 1, \textit{b} -- 2, \textit{c} -- "sample".

Кроме того, мы можем создать так называемый блок:

\lstinputlisting[language=go,style=lststyle:golang]{012-blocks.go}

ToDo

\begin{lstlisting}[language=go,style=lststyle:golang]
example := 3
a, b := 1, "sample
\end{lstlisting}

Переменные могут быть переопределены, а из значения можно присваивать другим переменным:
\lstinputlisting[language=go,style=lststyle:golang]{013-vars.go}

Обратите внимание, пременная существует внутри блока когда и всех дочерних блоков, за его пределами обратиться к ней невозможно. Имя уникально, внутри одной области видимости может существовать только одна переменная с таким идентификатором.

Переменные, которые были объявлены, но не получили значения, автоматически инициализируются со значением по умолчанию:
\begin{outline}
    \1 Для строки это пустая строка: "\,"
    \1 Для числа это 0
    \1 Для любых других типов данных \textit{nil}
\end{outline}

\end{document}