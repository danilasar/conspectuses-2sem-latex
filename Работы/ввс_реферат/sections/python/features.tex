Главная особенность синтаксиса Python заключается в том, что он использует отступы для отделения блоков кода, а не ставшие традиционными фигурные скобки. Эта особенность делает код менее нагруженным символами, но более читабельным. В целом Python имеет гораздо более простой синтаксис, чем многие другие языки программирования, поэтому он видится более доступным для начинающих\cite{HABR_ABOUT_PYTHON}.

Python поддерживает динамическую типизацию, что означает, что типы данных переменных не указываются явно. Это серьёзно усложняет отладку и отрицательно сказывается на производительности, но в то же время не только делает код гибче, но и упрощает его понимание человеком.

Важно отметить, что языку свойственна мультипарадигмальность. Python поддерживает различные парадигмы программирования, включая объектно-\\ориентированное, процедурное и функциональное.

Python предоставляет удобные структуры данных, такие как списки, словари и кортежи. Их легко создавать и ими просто манипулировать. Кроме того, в Python присутствуют так называемые генераторы, которые позволяют создавать итерируемые объекты, экономя при этом память и не занимая много времени для создания всех элементов сразу. В дополнение к этому Python поддерживает метапрограммирование, позволяющее писать код, который создаёт или модифицирует другой код во время выполнения.

Python является интерпретируемым языком, что означает, что его код не компилируется в машинный код до его выполнения.