\subsubsection{Веб-разработка}
Python широко используется в веб-разработке\cite{HABR_WHERE_PYTHON} благодаря своей универсальности, простому в освоении синтаксису и обширному набору библиотек, среди которых есть:
\begin{itemize}
    \item Django: Фреймворк для разработки веб-приложений, известный своей полнотой и безопасностью.
    \item Flask: Микрофреймворк для создания простых и легких веб-приложений.
    \item FastAPI: Высокопроизводительный фреймворк для создания RESTful API с использованием асинхронного программирования.
\end{itemize}

В целом, Python предоставляет разработчикам мощные и универсальные инструменты для создания широкого спектра веб-приложений, от простых веб-сайтов до сложных веб-сервисов и приложений электронной коммерции.


\subsubsection{Статистика и аналитика}

Python приобрёл широкую популярность в статистике, аналитике и машинном обучении (МО). Его простота, универсальность и обширная экосистема библиотек позволяют разработчику сконцентрироваться непосредственно на сложнейших аналитических и статистических вычислениях, не смешивая логику продукта с его низкоуровневой реализацией. Это обусловило появление множества библиотек для работы с большими данными, для анализа и для построения моделей, что, в свою очередь, сделало данный язык почти идеальным выбором для этой сферы.

Python имеет ряд инструментов для визуализации данных и отладки, что делает его удобным для исследовательской работы в области МО. Разработчики могут быстро исследовать данные, оценивать модели и получать наглядное представление о результатах.

Среди библиотек для аналитики данных можно выделить библиотеки: Pandas, Matplotlib, Seaborn и Power BI. Они упрощают и автоматизируют этапы аналитического процесса, позволяя аналитикам сосредоточиться на интерпретации результатов. С помощью Python генерируются информативные графики и диаграммы для наглядного представления данных, ведётся поиск закономерностей и аномалий в данных для выявления тенденций и прогнозирования.

Python стал незаменимым инструментом для статистиков, аналитиков и специалистов в области машинного обучения. Его простота использования, универсальность и обширная экосистема библиотеки делают его хорошим выбором для работы с большими данными, анализа и построения моделей.
