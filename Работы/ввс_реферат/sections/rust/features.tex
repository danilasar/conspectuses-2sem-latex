К главным особенностям языка Rust, делающим его уникальным, можно отнести:
\begin{itemize}
    \item Система типов: Rust отличается от других языков программирования своей передовой системой типов, которая гарантирует безопасность памяти во время компиляции, устраняя целые классы ошибок.
    \item Владение: Rust использует концепцию владения, которая отличается от традиционных систем управления памятью, таких как сборка мусора или подсчет ссылок. Она отслеживает, кто владеет данными в любой момент времени, предотвращая проблемы с одновременным доступом.
    \item Неизменяемые ссылки: Rust поддерживает неизменяемые ссылки, что гарантирует, что данные не могут быть изменены после их создания.
\end{itemize}

Благодаря этому Rust гарантирует безопасность памяти ещё во время компиляции, устраняя распространённые ошибки, такие как использование неинициализированных переменных или выход за пределы массива.

Кроме этого, Rust имеет и ряд других особенностей:
\begin{itemize}
    \item Rust является компилируемым языком, который генерирует высокоэффективный машинный код, что делает его подходящим для высокопроизводительных приложений.
    \item Rust поддерживает параллельное программирование через каналы и мьютексы, что облегчает создание многопоточных приложений.
    \item Rust предоставляет мощную систему макросов --- инструмент метапрограммирования, позволяющий создавать код, генерирующий или модифицирующий другой код.
\end{itemize}

Среди популярных инструментов и библиотек Rust можно выделить следующие:
\begin{itemize}
    \item Cargo --- менеджер пакетов для Rust, который упрощает установку, сборку и управление зависимостями.
    \item Serde --- библиотека для сериализации и десериализации данных в различных форматах, таких как JSON, XML и CSV.
    \item Actix Web --- фреймворк для создания высокопроизводительных и масштабируемых веб-приложений.
    \item Async-std --- асинхронная стандартная библиотека для Rust, которая обеспечивает асинхронные альтернативы стандартным библиотечным типам и функциям.
\end{itemize}

