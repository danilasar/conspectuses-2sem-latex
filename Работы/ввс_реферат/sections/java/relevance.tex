Java остаётся одним из самых популярных языков программирования в бизнес-сфере. Он обеспечивает надёжность, масштабируемость и удобство разработки, что делает его привлекательным выбором для крупных корпораций.

Переносимость кода и кроссплатформенность, высокая масштабируемость и обширное сообщество разработчиков делают этот язык отличным выбором для крупных корпоративных программ.

Популярность Java серьёзно ограничивают проблемы с производительностью, связанные с использованием виртуальной машины для исполнения программ, написанных на этом языке. Безусловно, JVM обязана быть установлена и развёрнута на любых устройствах, запускающих Java-программу, что также отрицательно сказывается на применимости языка в некоторых областях.

Хотя синтаксис Java прекрасно подходит для крупных проектов, он является слишком громоздким и неповоротливым для небольших программ, что заставляет разработчиков программного обеспечения такого масштаба делать выбор в пользу других языков.

В связи с тем, что Android переориентируется на Kotlin, Java также теряет свои позиции в мобильной разработке.

Java является мощным и универсальным языком программирования, подходящим для широкого круга задач. Несмотря на некоторые недостатки, его преимущества делают его незаменимым в различных областях разработки программного обеспечения.