Java --- один из самых популярных и широко используемых языков программирования, разработанный компанией Sun Microsystems (ныне Oracle Corporation) в 1995 году, известный своей платформонезависимостью, безопасностью и масштабируемостью. Он является универсальным инструментом для разработки приложений различного назначения\cite{JAVA}.

Немаловажным плюсом Java является встроенная поддержка многопоточности. В целом, стандартная библиотека этого языка весьма богата и имеющийся набор встроенных классов позволяет избежать использования слишком большого числа сторонних разработок.

Java использует принцип "write once, run anywhere" (WORA), что означает, что программы могут работать на любой платформе, где установлена виртуальная машина Java (JVM). Платформонезависимость достигается за счёт того, что компилятор Java транслирует исходники в байткод, являющийся промежуточным звеном между исходным и машинным кодами. Его, в свою очередь, при запуске на конечной машине интерпретирует JVM.

Java полностью построен на принципах объектно-ориентированного программирования, любой код в нём представлен в виде структуры классов. Безусловно, Java поддерживает инкапсуляцию, наследование и полиморфизм. всё это упрощает разработку и обслуживание крупных проектов.