Kotlin предлагает более лаконичный и понятный код по сравнению с Java, что упрощает разработку и поддержку кода и влияет на выбор разработчиков в пользу этого языка.

Кроме того, Google официально позиционирует Kotlin как основной инструмент разработки приложений для её детища в лице Android, что укрепляет позиции этого языка в мобильной разработке.

Примечательной особенностью Kotlin является возможность разрабатывать приложения не только для Android, но и для iOS. Эта возможность предоставляется таким средством разработки как Kotlin Multiplatform Mobile (KMM). Впрочем, здесь он сталкивается с серьёзной конкуренцией против Swift.

Таким образом, Kotlin представляется как серьёзный конкурент Java и активно набирает популярность. Его возможности и поддержка со стороны крупных корпораций делают его перспективным инструментом для будущего в мире программирования.