Исходя из вышеизложенного материала можно сделать вывод, что рассмотренные языки вместе покрывают подавляющее большинство сфер деятельности и для решения схожих задач существуют различные инструменты, каждый из которых обладает своими преимуществами и недостатками. Так,

\begin{itemize}
  \item В сфере веб-разработки широко распространены Go, Python и Java
  \item Для работы с большими данными чаще всего используются Python и Go
  \item Для разработки высокопроизводительных или работающих непосредственно с оборудованием программ, как правило, применяют Rust и C++
  \item В области машинного обучения весьма востребованы Python и C++
  \item При разработке мобильных приложений, как правило, останавливаются на Swift и Kotlin или Java
\end{itemize}

Безусловно, каждый язык имеет свои нюансы, делающие его более или менее благоприятным выбором для решеня тех или иных задач, но также немаловажным фактором в распространении языка в конкретных областях оказалось наличие не только встроенных инструмнтов, но и библиотек, поставляемых отдельно и упрощающих работу в данной сфере.

Не стоит отрицать и обратную зависимость, но факт состоит в том, что "современность" языка определяют не только его личные качества, но и наличие сформировавшегося вокруг него сообщества. Это также препятствует быстрому распространению новых инструментов, вокруг которых подобная экосистема только-только образовывается.
