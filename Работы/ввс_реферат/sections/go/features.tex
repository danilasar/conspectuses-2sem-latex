Язык программирования Go (или Golang) был разработан компанией Google и приспособлен для применения в различных областях. Это самобытный язык, получивший признание за простоту, эффективность и надёжность\cite{BOOK_ABOUT_GO}.

Синтаксис Go прост и лаконичен, что делает код более читаемым и поддерживаемым. Он включает в себя строгую типизацию, но также предоставляет удобные инструменты для работы с интерфейсами и конкурентным программированием.

Уникальный функционал Go включает в себя горутины (goroutines) и каналы (channels) для удобного параллельного выполнения задач. Это делает Go особенно подходящим для разработки высокопроизводительных и распредёленных систем. Кроме этого, отличительной особенностью данного языка является отложенного выполнения, суть которой заключается в том, что помеченный ключевым словом defer код выполняется в конце работы функции, а не в месте его фактического расположения. Это позволяет избегать множество ошибок в процессе разработки, чаще всего, но не только, связанных с контролем утечек памяти.

Также для безопасного и эффективного доступа к данным в языке Golang используются яркие указатели.
