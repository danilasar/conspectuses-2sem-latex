Поскольку C++ предоставляет низкоуровневый доступ к памяти, это делает его подходящим для написания высокопроизводительных приложений и операционных систем.

При этом C++ имеет множество библиотек, охватывающих широкий спектр областей применения, что снижает необходимость писать собственный код. Среди них есть такие общие и специализированные библиотеки как:
\begin{itemize}
    \item Boost --- коллекция библиотек C++, обеспечивающих общие функции и алгоритмы, такие как контейнеры, умные указатели и параллелизм.
    \item Eigen --- библиотека для линейной алгебры, которая предоставляет высокопроизводительные реализации матричных операций.
    \item Qt --- кроссплатформенный фреймворк для разработки графического интерфейса, который позволяет создавать приложения с богатым графическим интерфейсом.
    \item OpenCV --- библиотека для обработки изображений и компьютерного зрения, которая предоставляет широкий спектр функций для обработки изображений, анализа и распознавания образов.
\end{itemize}

Учитывая существующий инструментарий, плюсы и минусы C++, нетрудно определить основные сферы разработки, в которых применяется этот язык програмирования:
\begin{itemize}
    \item Разработка игр: C++ является популярным выбором для разработки высокопроизводительных игр благодаря своей скорости и графическим возможностям.
    \item Операционные системы: C++ используется в операционных системах, таких как Windows и Linux, для создания низкоуровневых компонентов и драйверов.
    \item Встраиваемые системы: C++ используется встраиваемых системах, таких как автомобильные контроллеры и медицинские устройства, где скорость и эффективность имеют решающее значение.
    \item Научные вычисления: C++ используется в научных вычислениях для создания высокопроизводительных симуляций и моделей.
    \item Финансовые технологии и финансовое моделирование: C++ используется в финансовом моделировании для создания сложных моделей и алгоритмов, требующих высокой скорости и точности.
    \item Промышленность и вооружённые силы: разработка критически важных систем, таких как систем автономного вождения, системы управления полётом и системы управления оружием.
    \item Машинное обучение: C++ остаётся популярным для разработки приложений искусственного интеллекта и машинного обучения. Его высокая производительность и возможности параллелизма делают его идеальным выбором для создания высокопроизводительных моделей и приложений машинного обучения.
\end{itemize}