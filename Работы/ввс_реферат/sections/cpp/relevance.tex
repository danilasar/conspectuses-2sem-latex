Есть несколько проблем, которые могут помешать дальнейшему развитию C++:
\begin{itemize}
    \item Крутая кривая обучения: C++ имеет крутую кривую обучения из-за его сложного синтаксиса и необходимости понимания управления памятью и указателей. Это может отпугнуть новых разработчиков и затруднить поиск и удержание квалифицированных разработчиков C++.
    \item Управление памятью: C++ требует ручного управления памятью, что, конечно, благоприятно сказывается на производительности в руках грамотного специалиста, но это же может привести к ошибкам и утечкам памяти. Отсутствие автоматического управления мусором может усложнить разработку и обслуживание приложений на C++.
    \item Отсутствие поддержки многопоточности на уровне языка: C++ изначально не поддерживает многопоточность на уровне языка. Хотя существуют сторонние библиотеки, которые предоставляют этот функционал, это может усложнить написание и отладку многопоточных приложений.
    \item Несовместимость между версиями: Различные версии C++ не всегда совместимы друг с другом, что может затруднить обновление кодовой базы до новых версий языка.
    \item Отсутствие единого стандарта: Существует несколько стандартов C++, что может привести к несовместимости между различными компиляторами и платформами.
\end{itemize}

И несмотря на то, что C++ является доминирующим языком в системном программировании и высокопроизводительных вычислениях, есть несколько потенциальных конкурентов,  которые могут бросить ему вызов:
\begin{itemize}
    \item Rust --- относительно новый язык, который был разработан с учётом безопасности и производительности. Он имеет возможности параллелизма и низкоуровневый контроль, что делает его привлекательным для системного программирования и высокопроизводительных вычислений.
    \item Go --- компилируемый язык, известный своей производительностью, простотой использования и возможностями параллелизма. Go также набирает популярность в области системного программирования и высокопроизводительных вычислений, но он не так зрелый, как C++.
    \item Ada --- язык программирования общего назначения, разработанный для критически важных применений. Ada имеет строгие правила безопасности и надежности, что делает его популярным в таких областях, как аэрокосмическая и оборонная промышленность.
\end{itemize}

Рост популярности других языков программирования, таких как Rust и Go, которые предлагают аналогичную производительность и более современные функции, может представлять собой проблему для дальнейшего развития C++.

Тем не менее, сообщество C++ активно работает над решением этих проблем. Например, в стандарте C++23 планируется ввести концепции, которые помогут улучшить безопасность и выразительность языка. Кроме того, существуют сторонние библиотеки, такие как Boost и Qt, которые предоставляют современные функции и упрощают разработку на C++.

В целом, хотя C++ остается мощным и востребованным языком, проблемы, упомянутые выше, могут стать препятствием для его дальнейшего развития. Однако сообщество C++ активно работает над решением этих проблем, и вполне вероятно, что C++ останется актуальным языком в обозримом будущем.