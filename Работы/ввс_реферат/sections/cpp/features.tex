Язык C++ имеет несколько особенностей:
\begin{itemize}
    \item Слабая статическая типизация: В отличие от динамически типизированных языков, таких как Python, C++ требует явного указания типов данных, что делает код более явным и менее подверженным ошибкам во время выполнения.
    \item Указатели: C++ предоставляет прямой доступ к указателям, что позволяет разработчикам напрямую управлять памятью, напрямую обращаться к адресам памяти и манипулировать ими. Таким образом можно создавать высокопроизводительный код.
    \item Перегрузка операторов: C++ позволяет перегружать операторы, что означает, что операторы могут иметь разное поведение для разных типов данных.
    \item Шаблоны: C++ поддерживает шаблоны, которые позволяют создавать обобщенный код, который может использоваться с различными типами данных.
\end{itemize}

Благодаря этим решениям C++ поддерживает различные парадигмы программирования, включая объектно-ориентированное, процедурное и функциональное.

Важно отметить, C++ является компилируемым языком и поддерживает работу на различных уровнях абстракции вплоть до низкоуровневого программирования. Это делает его очень быстрым и эффективным, поскольку позволяет писать высокопроизводительный код для решения сложнейших задач с минимальным расходом вычислительных мощностей.