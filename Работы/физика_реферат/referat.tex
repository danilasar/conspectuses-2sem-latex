\documentclass[referat,times]{SCWorks}
% Тип обучения (одно из значений):
%    bachelor   - бакалавриат (по умолчанию)
%    spec       - специальность
%    master     - магистратура
% Форма обучения (одно из значений):
%    och        - очное (по умолчанию)
%    zaoch      - заочное
% Тип работы (одно из значений):
%    coursework - курсовая работа (по умолчанию)
%    referat    - реферат
%  * otchet     - универсальный отчет
%  * nirjournal - журнал НИР
%  * digital    - итоговая работа для цифровой кафдры
%    diploma    - дипломная работа
%    pract      - отчет о научно-исследовательской работе
%    autoref    - автореферат выпускной работы
%    assignment - задание на выпускную квалификационную работу
%    review     - отзыв руководителя
%    critique   - рецензия на выпускную работу
% Включение шрифта
%    times      - включение шрифта Times New Roman (если установлен)
%                 по умолчанию выключен

\usepackage[T2A]{fontenc}
\usepackage[utf8]{inputenc}
\usepackage{graphicx}
\usepackage[sort,compress]{cite}
\usepackage{amsmath}
\usepackage{amssymb}
\usepackage{amsthm}
\usepackage{fancyvrb}
\usepackage{longtable}
\usepackage{array}
\usepackage[english,russian,greek]{babel}
\usepackage{minted}
\usepackage{tempora}
\usepackage[hidelinks]{hyperref}

\usepackage{multirow}
\usepackage[table]{xcolor}\usepackage{longtable}\usepackage{array}
\usepackage{graphicx}%Вставка картинок правильная

\usepackage{float}%"$\Pi$лавающие" картинки

\usepackage{wrapfig}%Обтекание фигур (таблиц, картинок и прочего)
\setlength{\arrayrulewidth}{0.5mm}
\setlength{\tabcolsep}{18pt}

\linespread{1}

\newenvironment{centeritemize*}[1][]
  {\par\centering\begin{itemize*}[itemjoin=\quad,#1]}
  {\end{itemize*}\par}
\newcolumntype{C}[1]{>{\centering\let\newline\\\arraybackslash\hspace{0pt}}m{#1}}

\usepackage{tikz}  % $\Gamma$рафы
\usetikzlibrary{graphs}
\usepackage{bchart} % Диаграммы раз
\usepackage{pgfplots} % Диаграммы дваз

\usepackage{graphicx}
\usepackage{chngcntr}


\begin{document}
\selectlanguage{russian}
\chair{математической кибернетики и компьютерных наук}
\title{Скорость звука в газовой среде (на основе модели идеального газа).}
\course{1}
\group{151}
\napravlenie{09.03.04 "--- $\Pi$рограммная инженерия}
\author{$\Gamma$ригорьева Данилы Евгеньевича}
% Научный руководитель (для реферата преподаватель проверяющий работу)
\satitle{доцент, к.\,ф.-м.\,н.} %должность, степень, звание
\saname{О.\,А.\,Черкассова}
% $\Gamma$од выполнения отчета
\date{2023}

\maketitle

% Включение нумерации рисунков, формул и таблиц по разделам (по умолчанию -
% нумерация сквозная) (допускается оба вида нумерации)
\secNumbering

\tableofcontents

\newpage

\intro
\par Скорость звука в газовой среде является одним из ключевых параметров, определяющих распространение звуковых волн и играющих важную роль в акустике и термодинамике. Изучение этого параметра позволяет не только понять основы звуковой теории, но и применить полученные знания на практике для решения различных задач.
\par В данном реферате мы подробно рассмотрим вывод формулы скорости звука в идеальном газе, основанный на модели идеального газа и законах термодинамики. Мы разъясним физические основы, лежащие в основе этой формулы, и продемонстрируем, как она используется для расчета скорости звука в различных газовых средах.
\par Кроме того, мы расскажем о методе интерференции звуковых колебаний, который позволяет определить скорость звука путем измерения разности фаз между звуковыми волнами. Этот метод находит применение в практических условиях и позволяет экспериментально подтвердить теоретические расчеты скорости звука.
\par Надеемся, что изучение данной темы поможет студентам технических специальностей и всем, кто интересуется акустикой и термодинамикой, углубить свои знания и применить их на практике для решения разнообразных задач в области звуковой физики.

\section{Звук}
\subsection{Понятие звука. Процесс распространения звука в газах}
\par \textbf{Звуковые колебания} -- колебания плотности газовой среды с частотой, находящейся в пределах от 16 до 20 000 $\Gamma$ц. Именно такие колебания воспринимаются человеческим ухом.
\par \textbf{Звуковые (акустические) волны} -- упругие волны малой интенсивности.
\par Элементы среды, примыкающие к источнику колебаний, смещаются из своих положений равновесия и совершают колебания по тому же закону, что и мембрана. Смещения происходят в направлении, перпендикулярном колеблющемуся телу. Благодаря упругим взаимодействиям между отдельными элементами среды в колебательное движение придут через определённое время и элементы, находящиеся от источника на некотором расстоянии. Образующаяся в газовой среде волна смещений является \textbf{продольной}, то есть колебание элементов среды около их положения равновесия происходит вдоль направления распространения волны, так как в газах деформации сдвига неупруги и, следовательно, поперечные волны не могут существовать. Волновые поверхности -- поверхности, на которых все элементы однородной среды совершают одинаковые движения -- представляют собой в данном случае плоскости, перпендикулярные направлению распространения волны. Такие волны называются \textbf{плоскими}.
\par В результате смещений элементов газовой среды из их положений равновесия вдоль направления распространения волны будут чередоваться области с повышенной и пониженной плотностью по сравнению с той, которая была до прихода волны. Соответственно вдоль направления распространения волны будет изменяться и направление газа. Распределение плотности вдоль направления распространения звуковой волны можно видеть на рисунке 1.1.
\par \textbf{Колебательная скорость} -- скорость колебательного движения упругой среды. В упругой среде колебания постепенно затухают. Для характеристеки затухающих колебаний используется коэффициент затухания ($S$), логарифмический декремент ($D$) и добротность ($Q$).
\par \textbf{Время затухания} $\tau$ -- время, в течение которого амплитуда уменьшается в $e$ раз.
\par \textbf{Коэффициент затухания} -- скорость убывания амплитуды с течением времени, обратно пропорциональная времени затухания:
\begin{equation*}
    S = \tau^{-1}
\end{equation*}
\par \textbf{Логарифмический декремент} -- уменьшение амплитуды за один цикл, характеризующийся отношением периода колебаний ко времени затухания.
\begin{equation*}
    D = \frac{T}{\tau}
\end{equation*}
$\Pi$ри воздействии на колебательную систему с периодической силой возникают вынужденные колебания, повторяющие изменения внешней силы. Частота вынужденных колебаний не зависит от параметров колебательной системы и совпадает с частотой собственных незатухающих колебаний механической системы.
\par \textbf{Амплитуда} -- явление, когда амплитуда колебательной скорости достигает максимального значения. В отличии от частоты, она зависит от массы, механического сопротивления и гибкости системы.
\par $\Pi$ри частотах воздействия, значительно меньших резонансной, внешняя гармоническая сила уравновешивается практически только силой упругости. $\Pi$ри частотах возбуждения, близких к резонансной, главную роль играют силы трения. $\Pi$ри условии, когда частота внешнего воздействия значительно больше резонансной, поведение колебательной системы зависит от силы инерции или массы.
\begin{figure}
    \textbf{\caption{Распределение плотности вдоль направления распространения звуковой волны}
  \label{fig:plotnost}}
    \includegraphics[width=\textwidth]{plotnost.png}
\end{figure}

\subsection{Звуковые колебания, не воспринимаемые человеческим ухом}
\par \textbf{Инфразвук} -- звуковые колебания низкой частоты. Человек способен слышать волны частотой от 16 $\Gamma$ц; инфразвуковые колебания не превышают этот показатель.
\par Инфразвук подчиняется тем же закономерностям, что и слышимый звук. Его особенность заключается в том, что он слабо поглощается средой и может распространяться на огромные расстояния.
\begin{itemize}
    \item Амплитуды колебаний инфразвука значительно превышают амплитуды слышимого звука
    \item Для инфразвука характерна дифракция
    \item Инфразвук вызывает вибрацию крупных объектов, так как входит в резонанс с ним. Эта его особенность позволяет использовать сильные инфразвуковые колебания в качестве поражающего средства
\end{itemize}
\par \textbf{Ультразвук} -- упругие звыуковые колебания высокой частоты. Человеческое ухо воспринимает распространяющиеся в среде упругие волны частотой приблизительно до 16-20 к$\Gamma$ц; колебания с более высокой частотой представляют собой ультразвук.\cite{ULTRAZVUK_1}
\subsection{Виды звука}
\begin{tikzpicture}
    %% Опция nodes определяет параметры всех узлов графа. align=center центрирует текст в узле
    \graph[nodes={align=center}, grow down sep, branch right sep] {
        "\textbf{Тон}\\Звук, представляющий собой периодический процесс" ->
        {
            "\textbf{$\Pi$ростой тон}\\Создаётся телом,\\колеблющимся по гармоническому закону",
            "\textbf{Сложный тон}\\Создаётся периодическими, но\\ не гармоническими колебаниями\\и может быть разложен как\\сумма простых тонов"
        } 
    };
\end{tikzpicture}
\textbf{Шум} -- звук, имеющий сложную неповторяющуюся временную зависимость и представляющий собой сочетание беспордочно изменяющихся сложных тонов.
\subsection{Физические параметры звука}
\par \textbf{Акустический спектр} --  совокупность всех звуковых частот тона с указанием их относительных интенсивностей или амплитуд. Наименьшая частота в спектре ($\nu$) соответствует основному тону, а остальные частоты называют обертонами (частотами, кратными основной частоте) или гармониками (остальными частотами). Акустический спект задаётся функцией частоты $\nu$ или длины волны $\lambda$ и выражает относительную роль разных частот или длин упругих волн в излучаемом звук. На рисунке \ref{fig:royal} показан спектр ноты $\nu = 100 \, \text{$\Gamma$ц}$, взятой на рояле:

\begin{figure}[h!]
    \textbf{\caption{Спектр ноты рояля}
  \label{fig:royal}}
\begin{tikzpicture}[baseline=(current  bounding  box.north)]
\begin{axis}[
        bar width = 0.1,xlabel={$\nu$ [$\Gamma$ц]}, ylabel={$A$ [\%]},
        ymajorgrids=true,
        xmajorgrids=true,
        ybar=0pt
      ]
    \addplot coordinates {(150,100)
    (250,60)
    (359,30)
    (500,10)
    (600,51)
    (700,10)
    (850,15)
    (1000,15)
    (1100,10)(1200,10)(1700,10)(1800,10)
    };
    \end{axis}
\end{tikzpicture}
\end{figure}
\par \textbf{Звуковое давление} $\Delta P$ -- амплитуда тех изменений давлений в среде, которые возникают при прохождении звуковой волны.
\par \textbf{Интенсивность звука} $I$ -- скалярная физическая величина, характеризующая мощность, переносимую в направлении распространения звука. В однородной среде интенсивность звука, испущенного в данном направлении, убывает по мере удаления от источника звука. $\Pi$ри использовании волноводов можно добиться и увеличения интенсивности. Связь между интенсивностью $I$ и звуковым давлением $\Delta P$ выражается следующей формулой:
\begin{equation}
    I = \frac{\Delta P^2}{2 \rho v},
\end{equation}
где:
\begin{itemize}
    \item $I$ -- интенсивность звука,
    \item $\Delta P$ -- звуковое давление,
    \item $\rho$ -- плотность среды,
    \item $v$ -- скорость звука в ней.
\end{itemize}
\par \textbf{Уровень интенсивности} -- десятичный логарифм отношения интенсивности звука к порогу слышимости (минимальному значению звукового давления и интенсивности звука, при которых у человека возникают слуховые ощущения):
\begin{equation}
    L_\text{Б} = \lg{\frac{I}{I_0}} = 2 \lg{\frac{\Delta P}{\Delta P_0}},
\end{equation}
где:
\begin{itemize}
    \item $L_\text{Б}$ -- уровень интенсивности в белах,
    \item $I$ -- интенсивность звука,
    \item $\Delta P$ -- звуковое давление,
    \item $I_0$ -- минимальное значение интенсивности звука,
    \item $\Delta P_0$ -- минимальное значение звукового давления.
\end{itemize}
\section{Скорость звука в идеальном газе}
Скорость распространения продольных волн в упругой среде определяется отношением
\begin{equation}
    v = \sqrt{\frac{B}{\rho}} = \sqrt{\frac E \rho},
\end{equation}
где
\begin{itemize}
    \item $v$ -- скоость распространения продольных волн,
    \item $B$ -- модуль всестороннего сжатия,
    \item $E$ -- модуль Юнга (упругости среды). Модуль Юнга зависит только от свойств материала и не зависит от размеров и формы тела.
    \item $\rho$ -- плотность среды.
\end{itemize}
\par Ещё Ньютон пытался вычислить значение скорости звука в воздухе. Он предположил, что упругость воздуха равна атмосферному давлению $P_{атм}$. Французский учёный $\Pi$ьер Симон маркиз де Лаплас показал, что предположение Ньютона равносильно предположению о быстром выравнивании температуры между областями разрежения и сжатия. Эта гипотеза может не выполняться в случае плохой теплопроводности газа и малого периода колебаний в звуковой волне. На самом деле между областями разрежения и сжатия газа возникает разность температур, которая существенно влияет на упругие свойства. Лаплас предположил, что сжатие и разрежение газа в звуковой волне происходят по адиабатическому закону, то есть вовсе без влияния теплопроводности. Формула Лапласа, выведенная в 1816 году, имеет вид:
\begin{equation}
    v = \sqrt{\frac{\gamma p}{\rho}},
\end{equation}
где:
\begin{itemize}
    \item $p$ -- среднее давление в газе,
    \item $\rho$ -- средняя плотность,
    \item $\gamma$ -- константа, зависящая от свойств конкретного газа.
\end{itemize}
\par Значение $\gamma$ рассчитывается по следующей формуле:
\begin{equation}
    \gamma = \frac{C_P}{C_V},
\end{equation}
где:
\begin{itemize}
    \item $C_p$ -- молярная теплоёмкость в изобарном процессе (теплоёмкость процесса с постоянным давлением, т. е. при $p \equiv const$),
    \item $C_V$ -- молярная теплоёмкость в изохорном процессе (теплоёмкость процесса с постоянным объёмом, т. е. при $V \equiv const$)
\end{itemize}
\par $\Pi$одставив значение $\gamma$ в формулу Лапласа, мы получим следующее выражение:
\begin{equation}
    v = \sqrt{\frac{C_p p}{C_V \rho}}
\end{equation}
\par Воспользуемся равенством
\begin{equation}
    C_p = \frac{Q_p}{\Delta T} = C_V + p \frac{\Delta V}{\Delta T},
\end{equation}
где:
\begin{itemize}
    \item $Q_p$ -- количество теплоты,
    \item $\Delta V$ -- изменение объёма 1 моля идеального газа при изменении его температуры на $\Delta T$.
\end{itemize}
\par и подставим его в формулу (2.4):
\begin{equation}
    v = \sqrt{\frac{(C_V + p \frac{\Delta V}{\Delta T}) p}{C_V \rho}}
\end{equation}
\par В 1834 году французский физик Бенуа $\Pi$оль Эмиль Клайперон\cite{KLAYPERON_BIO}, исследуя цикл Никола Леонар Сади Карно, установил зависимость между давлением, молярным объёмом и абсолютной температурой идеального газа. Согласно этой зависимости,
\begin{equation}
    \frac{\rho V}{T} \equiv const,
\end{equation}
где
\begin{itemize}
    \item $\rho$ -- давление,
    \item $v$ -- количество вещества в молях,
    \item $T$ -- абсолютная температура.
\end{itemize}
что можно записать в виде
\begin{equation}
    \rho V = r T,
\end{equation}
где:
\begin{itemize}
    \item $\rho$ -- давление,
    \item $v$ -- количество вещества в молях,
    \item $T$ -- абсолютная температура,
    \item $r$ -- некая константа, которую следует вычислять для каждого газа.
\end{itemize}
Универсальная газовая постоянная $R$ -- фундаментальная физическая постоянная, входящая в уравнение состояния 1 моля идеального газа.
В 1874 году русский доктор химии Дмитрий Иванович Менделеев обратил внимание на прямую связь значения $r$ с количеством вещества\cite{MENDELEEV_UPRUGOST_TOCHNOST}, которую можно выразить следующей формулой:
\begin{equation}
    r = \nu k N_A,
\end{equation}
где:
\begin{itemize}
    \item $\nu$ -- количество вещества в молях,
    \item $k$ -- постоянная Больцмана,
    \item $N_A$ -- постоянная Авогадро
\end{itemize}
В результате подстановки этой формулы в была выведена формула Клайперона -- Менделеева:
\begin{equation}
    \rho V = \nu R T,
\end{equation}
где $\;R = k N_A\;$ есть \textbf{универсальная газовая постоянная}. Данная константа равняется работе расширения 1 моля идеального газа под постоянным давлением при нагревании на 1 К\cite{SIVUHIN_PHYSICS}. Также её смысл может означать разность молярных теплоёмкостей при постоянном давлении и постоянном объёме. На текущий момент времени актуальным и точным считается значение
\begin{equation}
    R = 8,314\,462\,618\,153\,24 \; \frac{\text{Дж}}{\text{моль} \cdot \text{K}},
\end{equation}
вычисленное на основе значений постоянных Авогардо и Больцмана, точно зафиксированных в международной системе единиц (СИ) решением $\Gamma$енеральной конвенции по мерам и весам от 16 ноября 2018 года\cite{CGPM_2018}, которое вступило в силу 20 мая 2019 года. В соответствии с ним, постоянная Авогардо равна:
\begin{equation}
    N_A = 6,022\,140\, 76 \cdot 10^{23} \; \text{моль}^{-1}
\end{equation}
\par А постоянная Больцмана:
\begin{equation}
    k = 1,380\, 649 \cdot 10^{-23} \; \frac{\text{Дж}}{\text{К}}
\end{equation}
\par Вспомним выведенную выше формулу (2.6):
\begin{equation*}
    v = \sqrt{\frac{(C_V + p \frac{\Delta V}{\Delta T}) p}{C_V \rho}}
\end{equation*}
\par Отношение изменения объёма $\Delta V$ при изменении его температуры на $\Delta T$ может быть найдено из приведённого выше уравнения состояния идеального газа, если взять $T = \Delta T$, $V = \Delta V$, $\nu = 1$ моль:
\begin{equation}
    p \Delta V = R \Delta T
\end{equation}
\begin{equation}
    \frac{\Delta V}{\Delta T} = \frac{R}{p}
\end{equation}
\par Выходит:
\begin{equation}
    v = \sqrt{\frac{(C_V + R)p}{C_V \rho}}
\end{equation}
\par Таким образом, воспользовавшись уравнением состояния идеального газа, упростив, можно представить формулу Лапласа в следующем виде:
\begin{equation}
    c = \sqrt{\frac{\gamma R T}M} = \sqrt{\frac{\gamma k T}\mu},
\end{equation}
где:
\begin{itemize}
    \item $\gamma$ -- показатель адиабаты,
    \item $M$ -- масса,
    \item $\mu$ -- молярная масса,
    \item $T$ -- абсолютная температура,
    \item $R$ -- универсальная газовая постоянная.
\end{itemize}
\par $\Pi$олученная формула является формулой \textbf{скорости звука в идеальном газе}.
\subsection{Характеристики, от которых зависит скорость звука в идеальном газе}
\par Заметим, что согласно формуле (2.17) фазовая скорость в газе зависит от сорта газа ($\gamma$, $\mu$) и от его термодинамического состояния ($T$).
\par Формулу (2.2) можно также представить в другом виде, опираясь на закон
\begin{equation}
    \rho = \frac{\rho_0}{1 + \alpha t},
\end{equation}
где
\begin{itemize}
    \item $\rho$ -- плотность среды при $t^\circ C$,
    \item $\rho_0$ -- плотность при $0^\circ C$
    \item $\alpha$ -- коэффициент расширения газа (например, для воздуха это $0,004 \text{град}^{-1}$)
\end{itemize}
следующим образом:
\begin{equation}
    v = \sqrt{\frac{\gamma p_0(1 + \alpha t)}{\rho_0}},
\end{equation}
где:
\begin{itemize}
    \item $\gamma$ -- отношение удельной теплоёмкости при постоянном давлении $C_p$ к удельной теплоёмкости при постоянном объёме $C_V$,
    \item $\rho_0$ -- нормальное атмосферное давление,
    \item $p_0$ -- плотность при $0^\circ C$,
    \item $\alpha$ -- коэффициент расширения газа,
    \item $v_0$ -- скорость звука при $0^\circ$.
\end{itemize}
\par Это позволяет сделать вывод, что скорость звука не зависит от давления в газе, но зато она существенно зависит от температуры среды.

\section{Явление интерференции звуковых колебаний}
\subsection{Когерентные волны}
\par Если в среде распространяется одновременно несколько волн, то смещение частиц среды от положения равновесия равно сумме смещений, вызываемой каждой из волн по отдельности. Это утверждение называется принципом суперпозиции (наложения) волн.
\par В случае, когда колебания, обусловленные отдельными волнами в каждой из точек среды, обладают одинаковой частотой и постоянной разностью фаз, волны называются когерентными. При сложении когерентных волн возникает явление интерференции, заключающееся в том, что колебания в одних точках усиливают, а в других точках ослабляют друг друга. 
\par В идеальном газе интерференция звуковых волн проявляется через изменение давления и плотности газа в результате наложения волн. Когда в среде распространяются две звуковые волны, смещение её частиц от положения равновесия равно сумме смешений, вызываемой каждой из волн по отдельности. В этом заключается смысл принципа наложения (суперпозиции) волн.
\par Когда волны пересекаются, то есть накладываются друг на друга, и возникающие колебания обладают одинаковой частотой и постоянной разностью фаз, волны называются когерентными.
\par Они могут усилить друг друга (конструктивная интерференция), что приводит к увеличению давления и плотности в этой области. Или же они могут ослабить друг друга (деструктивная интерференция), что приводит к уменьшению давления и плотности. Это влияет на распространение звука в газе и может создавать зоны усиления и ослабления звуковых сигналов.
\subsection{Разность фаз волн. Разность хода}
\par Пусть в данную точку пространства приходят две волны, полученные от одного источника, но прошедшие до этой точки различные пути $x_1$ и $x_2$ соответственно. Уравнения этих волн можно представить в виде:
\begin{equation}
    \xi_1 = A_1 \sin{2\pi\bigg(\frac t T - \frac{x_1}{\lambda}\bigg)}
\end{equation}
\begin{equation}
    \xi_2 = A_2 \sin{2\pi\bigg(\frac t T - \frac{x_2}{\lambda}\bigg)},
\end{equation}
где:
\begin{itemize}
    \item $\xi_1$ -- смещение центра масс элемента газовой среды для первой волны,
    \item $\xi_2$ -- смещение для второй волны,
    \item $A_1$ -- амплитудное значение смещение центра масс элемента газовой среды для первой волны,
    \item $A_2$ -- амплитудное значение смещение центра масс элемента газовой среды для второй волны,
    \item $t$ -- момент времени,
    \item $T$ -- период колебания, равный времени, за который совершается одно полное колебание,
    \item $x_1$, $x_2$ -- пути первой и второй волн соответственно,
    \item $\lambda$ -- длина волны -- расстояние, на которое распространяется колебание за время, равное одному периоду колебания.
\end{itemize}
\par Рассмотрим также формулу фазы волны:
\begin{equation}
    \phi_i = 2\pi \frac{x_i}{\lambda},
\end{equation}
где $i \in \{1, 2\}$ -- номер волны, и выведем формулу разности фаз волн в данной точке:
\begin{equation}
    \phi_1 - \phi_2 = 2 \pi \frac{x_2 - x_1}{\lambda},
\end{equation}
где:
\begin{itemize}
    \item $\phi_1$, $\phi_2$ -- фазы волн в данной точке,
    \item $x_2 - x_1$ -- \textbf{разность хода}
\end{itemize}
\subsection{Зависимость интенсивности звука от разности хода интерферирующих волн}
\par Заинтересованному читателю нетрудно догадаться, что результирующая волна будет иметь ту же частоту $\omega$, а амплитуду и её начальную фазу можно определить следующими отношениями:
\begin{equation}
    A^2 = A_1^2 + A_2^2 + 2 A_1 A_2 \cos{2\pi \frac{x_2 - x_1}{\lambda}}
\end{equation}
и
\begin{equation}
    tg \, \phi = \frac{A_1 \sin{\phi_1} + A_2 \sin{\phi_2}}{A_1 \cos{\phi_1} + A_2 \cos{\phi_2}}
\end{equation}
\par Так как интенсивность звука $I$ прямо пропорциональна квадрату амплитуды звуковой волны $A$, из формулы (3.5) следует, что интенсивность звука также будет зависеть от разности хода нтерферирующих волн. Рассмотрим, при какой разности хода происходит усиление звука в данной точке пространства и при какой -- звуковые волны будут гасить друг друга.
\subsubsection{Случай, когда $A_1 = A_2$}
\par Из формулы (3.5), вычисляющий квадрат амплитуды, путём упрощения мы получим следующее выражение:
\begin{equation}
    A^2 = A^2_1\bigg(1 + \cos{2\pi\frac{x_2 - x_1}{\lambda}}\bigg)
\end{equation}
\par Достаточно легко заметить невооружённым глазом, что максимум интенсивности звука соответствует разности фаз
\begin{equation*}
    \phi_1 - \phi_2 = 0, \pm 2\pi, \pm 4 \pi, \dots
\end{equation*}
или разности хода
\begin{equation*}
    x_2 - x_1 = 0, \lambda, 2\lambda, 3\lambda, \dots .
\end{equation*}
\par При этом выполняется равенство
\begin{equation*}
    A^2 = 4 A_1^2,
\end{equation*}
то есть интенсивность звука в максимуме равна учетверённой интенсивности звука для каждого из интерферирующих колебаний.
\par Минимум интенсивности звука соответствует разности фаз
\begin{equation*}
    \phi_1 - \phi_2 = \pm \pi, \pm 3 \pi, \pm 5 \pi, \dots
\end{equation*}
или разности хода
\begin{equation*}
    x_2 - x_1 = \pm \frac \lambda 2, \pm 3 \frac \lambda 2, \pm 5 \frac \lambda 2 , \dots
\end{equation*}
\par При этом в минимуме наблюдается полное гашение звука.
\subsubsection{Случай, когда $A_1 \neq A_2$}
\par Тогда
\begin{equation}
    A^2 = A_1^2 + A_2^2 + 2A_1 A_2 \cos{2\pi \frac{x_2 - x_1}{\lambda}}
\end{equation}
\par Из этого выражения видно, что максимумы и минимумы интенсивности результирующей волны наблюдаются при тех же значениях разности хода между волнами, что и в случае равенства амплитуд интерферирующих волн.
\subsection{Определение скорости звука в идеальном газе методом интерференции}
\par Пусть разность хода двух волн, при которой достигается минимум интенсивности звука, равна $a$. Как следует из описанного выше, $a = \frac \lambda 2$, то есть
\begin{equation}
    \lambda = 4 a
\end{equation}
\par Вспомним, что скорость звука $c$ связана с длиной волны $\lambda$ и частотой $\nu$ следующим соотношением:
\begin{equation}
    v = 4 a \nu
\end{equation}
\par В таком случае, зная значения $a$ и $\nu$, мы можем без труда вычислить скорость звука $v$ в данной среде при данной температуре. Применив формулу
\begin{equation}
    c = \frac{v}{\sqrt{1 + \alpha t}},
\end{equation} мы можем установить скорость звука и при нулевой температуре.

\section{Интерференция колебаний, распространяющихся навстречу друг другу}
\subsection{Понятие стоячей волны}
\par Когда две плоские волны с одинаковой амплитудой и частотой движутся навстречу друг другу и налагаются, возникает колебательный процесс, называющийся \textbf{стоячей волной}. Это происходит из-за интерференции волн, при которой участки смещения волн совпадают во времени, что приводит к образованию узлов (мест, где колебания усиливают друг друга) и пучностей (мест, где колебания уничтожают друг друга).
\par Практически стоячие волны возникают при отражении волн от преграды, где падающая волна и отражённая волна, движущаяся навстречу друг другу, налагаются друг на друга, образуя стоячую волну. Это явление имеет важное значение, так как позволяет понять поведение волн и их взаимодействие, а также находит применение в различных областях, включая акустику, оптику и электромагнетизм.
\par В отличие от бегущей волны все точки, расположенные между двумя соседними узлами стоячей волны, колеблются в одной и той же фазе, но амплитуды их колебаний различны.
\par Стоячая волна не переносит энергии. Это объясняется тем, что стоячая волна является результатом сложения двух волн, переносящих равные энергии в противоположных направлениях.
\subsection{Определение скорости звука в воздухе методом стоячей волны}
\par Рассмотрим распространение звуковой волны в цилиндрической трубке длины $l$, на входе которой закреплена мембрана, колебания которой соответствуют гармоническому закону. Естественно, колеблясь, она вызывает колебания плотности воздуха в трубке. Другой конец трубки закрыт поршнем.
\par В данной конструкции в противоположные стороны будут двигаться звуковые волны, возбуждаемые мембраной и отражаемые поршнем, причём волна, движущаяся от мембраны к поршню, проходит путь $x$, а отражённая -- $2l - x$, где $l$ -- длина трубки. Соответственно, фазы волн будут различны. Учитывая, что при отражении от более плотной реды происходит изменение фазы отражённой волны на $\pi$, уравнения бегущей и отражённой волн можно записать соответственно в виде
\begin{equation}
    \xi_1 = A \sin{\omega\bigg(t - \frac x v\bigg)},
\end{equation}
\begin{equation}
    \xi_2 = A \sin \bigg[\omega\bigg(t - \frac{2t - x}{v}\bigg) - \pi\bigg] = - A \sin{\omega\bigg(t - \frac{2 l - x}{v}\bigg)},
\end{equation}
где:
\begin{itemize}
    \item $\xi_1$, $\xi_2$ -- смещения бегущей и отражённой волн соответственно,
    \item $A$ -- амплитуда обеих волн,
    \item $\omega=\frac{2\pi}{T}$ -- частота колебаний,
    \item $t$ -- момент времени,
    \item $l$ -- длина трубки,
    \item $x$ -- путь, пройденный бегущей волной,
    \item $2l - x$ -- путь, пройденный отражённой волной
\end{itemize}
\par Соответственно, элементы среды в сечении $x$ будут находиться в движении, определяемом суммой этих колебаний:
\begin{equation}
    \xi = \xi_1 = \xi_2 = 2A \sin{\omega \frac{l - x}{v}} \cos{\omega\bigg(t - \frac 1 v\bigg)}.
\end{equation}
\par Данная формула описывает стоячую волну, причём величина
\begin{equation}
    \bigg|2A \sin \omega \frac{l - x}{v}\bigg|
\end{equation}
называется амплитудой стоячей волны. Эта амплитуда для различных сечений $x$ будет различна, но для данного счения амплитуда остаётся одной и той же. Максимальное значение амплитуды колебания равно $2A$, минимальное -- нулю $\Rightarrow$ одни точки стоячей волны всё время колеблются с амплитудой, не превышающей максимум в $2A$, другие -- постоянно сохраняют состояние покоя. Положения первых точек (максимум амплитуды) соответствуют пучностям стоячей волны смещений, а положения вторых точек (амплитуда равна нулю) -- узлам стоячей волны смещений.
\par Положение пучностей и узлов в стоячей волне можно найти, полагая
\begin{equation*}
    \sin{\omega\frac{l - x}{v}} = \pm 1
\end{equation*}
и
\begin{equation*}
    \sin{\omega\frac{l - x}{v}} = 0
\end{equation*}
соответственно. Отсюда, учитывая, что
\begin{equation*}
    \lambda = vT,
\end{equation*}
получим
\begin{itemize}
    \item для пучностей:
\begin{equation}
    2\pi \frac{l - x}{\lambda} = (2n + 1) \frac \pi 2,
\end{equation}
    \item для узлов:
\begin{equation}
    2 \pi \frac{l - x}{\lambda} = 2n \frac \pi 2, \, \text{где} \, n = 0, 1, 2, \dots
\end{equation}
\end{itemize}
Следовательно, расстояния от поршня до пучностей и узлов будут иметь следующий вид:
\begin{equation}
    (l - x)_\text{пучн} = \frac \lambda 4, 3 \frac \lambda 4, 5 \frac \lambda 4, \dots,
\end{equation}
\begin{equation}
    (l - x)_\text{узл} = 0, \frac \lambda 2, 2 \frac \lambda 2, 3 \frac \lambda 2, \dots .
\end{equation}
\par Расстояние между двумя соседними пучностями или узлами равно $\frac \lambda 2$, а узлы и пучности находятся друг от друга на расстоянии, равном $\frac \lambda 4$.
\par Полагая, что в формуле, описывающей стоячую волну (4.3), значение функции $\cos{\omega\bigg(t - \frac l v\bigg)}$ равно нулю, можно найти моменты времени $t'_n$, в которые элементы среды проходят через свои положения равновесия. При этом имеем
\begin{equation}
    2\pi(\frac t T - \frac l \lambda) = (2n + 1) \frac \pi 2
\end{equation}
и
\begin{equation}
    t'_n = \frac l \lambda T + \frac{2n+1}{4} T,
\end{equation}
где
\begin{itemize}
    \item $T$ -- период колебания,
    \item $\lambda$ -- длига волны,
    \item $n = 0, 1, 2, \dots$.
\end{itemize}
\par Последнее выражение показывает, чо все элементы среды ппреодолевают свои положения равновесия одновременно ($t'_n$ не зависит от координаты $x$). Моменты времени $t'_n$ и $t'_{n+1}$ отделены друг от друга промежутками времени, равными $\frac T 2$.
\par Таким образом, в отличие от бегущей волны все точки, расположенные между двумя соседними узлами стоячей волны, колеблются в одной фазе, но амплитуды их колебаний различны.
\par Если частота колебания мембраны совпадает с частотой одного из собственных колебаний столба воздуха, наблюдается резонанс, отмечаемый по резкому усилению воздуха в трубке. Наименьшая разность длин двух столбов воздуха, в которых возникает резонанс, равна $\frac \lambda 2$.
\par Установив поршень в положение $l_1$, соответствующее первому максимуму громкости звука, ипределив положения $n+1$ максимума громкости звука $l_{n+1}$ в данной трубке, мы сможем сделать вывод, что на расстоянии $l_{n+1} - l_1$ должно укладываться $n$ полуволн $\frac \lambda 2$ и, следовательно,
\begin{equation}
    \lambda = 2 \frac{l_{n+2} - l_1}{n},
\end{equation}
где
\begin{itemize}
    \item $l_{n+1}$, $l_1$ -- положения первого и $n+1$-го максимумов,
    \item $n$ -- число полуволн на расстоянии $l_{n+1} - l_1$,
    \item $\lambda$ -- длина волны.
\end{itemize}
\par Тогда скорость звука в данной среде, в соответствии с формулой
\begin{equation*}
    v = \lambda \nu,
\end{equation*}
мы можем записать в виде выражения следующего вида:
\begin{equation}
    v = \lambda \nu = 2 \frac{l_{n+1} - l_1}{n} \nu,
\end{equation}
где
\begin{itemize}
    \item $v$ -- скорость звука в заданной среде при заданной температуре,
    \item $\nu$ -- частота колебаний мембраны,
    \item $\lambda$ -- длина волны,
    \item $n$ -- число полуволн на расстоянии $l_{n+1} - l_1$.
\end{itemize}
\par Вспомнив формулу
\begin{equation*}
    c = \frac{v}{\sqrt{1 + \alpha t}},
\end{equation*}
мы можем установить скорость звука и при нулевой температуре.

\conclusion
\par В ходе изучения скорости звука в газовой среде на основе модели идеального газа были рассмотрены основные принципы и выводы, лежащие в основе формулы скорости звука. Этот реферат позволяет глубже понять физические законы, определяющие распространение звука в газах, и применение этих знаний для вычисления скорости звука методом интерференции звуковых колебаний.
\par Полученные результаты представляют ценный материал и могут быть полезны как для студентов и исследователей в области технических наук, так и для всех, кто интересуется акустикой и термодинамикой. В целом, изучение скорости звука в газовой среде является важным шагом к пониманию физических явлений и их приложений в различных областях науки и техники.

\newpage

% Библиографический список, составленный вручную, без использования BibTeX
%
% \begin{thebibliography}{99}
%   \bibitem{Ione} Источник 1.
%   \bibitem{Itwo} Источник 2
%\end{thebibliography}

% Отобразить все источники. Даже те, на которые нет ссылок.
% \nocite{*}

\inputencoding{cp1251}
\bibliographystyle{gost780uv}
\bibliography{thesis}
\inputencoding{utf8}

% $\Pi$ри использовании biblatex вместо bibtex
% \printbibliography

% Окончание основного документа и начало приложений Каждая последующая секция
% документа будет являться приложением
\appendix

\end{document}