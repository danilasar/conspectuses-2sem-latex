\documentclass[otchet]{SCWorks}
% Тип обучения (одно из значений):
%    bachelor   - бакалавриат (по умолчанию)
%    spec       - специальность
%    master     - магистратура
% Форма обучения (одно из значений):
%    och        - очное (по умолчанию)
%    zaoch      - заочное
% Тип работы (одно из значений):
%    coursework - курсовая работа (по умолчанию)
%    referat    - реферат
%  * otchet     - универсальный отчет
%  * nirjournal - журнал НИР
%  * digital    - итоговая работа для цифровой кафдры
%    diploma    - дипломная работа
%    pract      - отчет о научно-исследовательской работе
%    autoref    - автореферат выпускной работы
%    assignment - задание на выпускную квалификационную работу
%    review     - отзыв руководителя
%    critique   - рецензия на выпускную работу
% Включение шрифта
%    times      - включение шрифта Times New Roman (если установлен)
%                 по умолчанию выключен
\usepackage[T2A]{fontenc}
\usepackage[utf8]{inputenc}
\usepackage{graphicx}
\usepackage[sort,compress]{cite}
\usepackage{amsmath}
\usepackage{amssymb}
\usepackage{amsthm}
\usepackage{fancyvrb}
\usepackage{longtable}
\usepackage{array}
\usepackage[english,russian]{babel}
\usepackage{minted}
\usepackage{tempora}
\usepackage[hidelinks]{hyperref}


\usepackage{multirow}
\usepackage[table]{xcolor}\usepackage{longtable}\usepackage{array}
\usepackage{graphicx}%Вставка картинок правильная

\usepackage{float}%"Плавающие" картинки

\usepackage{wrapfig}%Обтекание фигур (таблиц, картинок и прочего)
\setlength{\arrayrulewidth}{0.5mm}
\setlength{\tabcolsep}{18pt}

\newenvironment{centeritemize*}[1][]
  {\par\centering\begin{itemize*}[itemjoin=\quad,#1]}
  {\end{itemize*}\par}
\newcolumntype{C}[1]{>{\centering\let\newline\\\arraybackslash\hspace{0pt}}m{#1}}


\begin{document}

% Кафедра (в родительном падеже)
\chair{математической кибернетики и компьютерных наук}

% Тема работы
\title{Измерение моментов инерции тел}

% Курс
\course{1}

% Группа
\group{151}

% Факультет (в родительном падеже) (по умолчанию "факультета КНиИТ")
% \department{факультета КНиИТ}

% Специальность/направление код - наименование
% \napravlenie{02.03.02 "--- Фундаментальная информатика и информационные технологии}
% \napravlenie{02.03.01 "--- Математическое обеспечение и администрирование информационных систем}
% \napravlenie{09.03.01 "--- Информатика и вычислительная техника}
\napravlenie{09.03.04 "--- Программная инженерия}
% \napravlenie{10.05.01 "--- Компьютерная безопасность}

% Для студентки. Для работы студента следующая команда не нужна.
\studenttitle{Студентов}

% Фамилия, имя, отчество в родительном падеже
\author{Григорьева Данилы Евгеньевича и Лазаревой Виктории Владимировны}

% Заведующий кафедрой 
\chtitle{доцент, к.\,ф.-м.\,н.}
\chname{С.\,В.\,Миронов}

% Руководитель ДПП ПП для цифровой кафедры (перекрывает заведующего кафедры)
% \chpretitle{
%     заведующий кафедрой математических основ информатики и олимпиадного\\
%     программирования на базе МАОУ <<Ф"=Т лицей №1>>
% }
% \chtitle{г. Саратов, к.\,ф.-м.\,н., доцент}
% \chname{Кондратова\, Ю.\,Н.}

% Научный руководитель (для реферата преподаватель проверяющий работу)
\satitle{доцент, к.\,ф.-м.\,н.} %должность, степень, звание
\saname{О.\,А.\,Черкассова}

% Руководитель практики от организации (руководитель для цифровой кафедры)
\patitle{доцент, к.\,ф.-м.\,н.}
\paname{С.\,В.\,Миронов}

% Руководитель НИР
\nirtitle{доцент, к.\,п.\,н.} % степень, звание
\nirname{В.\,А.\,Векслер}

% Семестр (только для практики, для остальных типов работ не используется)
\term{2}

% Наименование практики (только для практики, для остальных типов работ не
% используется)
\practtype{учебная}

% Продолжительность практики (количество недель) (только для практики, для
% остальных типов работ не используется)
\duration{2}

% Даты начала и окончания практики (только для практики, для остальных типов
% работ не используется)
\practStart{01.07.2022}
\practFinish{13.01.2023}

% Год выполнения отчета
\date{2023}

\maketitle

\secNumbering

\newpage

\begin{enumerate}[leftmargin=3.5cm]
    \item[\bf Наименование работы.] Проверка основного уравнения вращательного движения с помощью маятника Обербека
    \item[\bf Цель.] Опытная проверка основного уравнения вращательного движения, оценка точности метода измерения.
    \item[\bf Принадлежности.] Маятник Обербека, секундомер, линейка, набор гирь и развесок.
    \item[\bf Рабочая формула.] $I = \frac m 2 \left( L^2 + \frac{d_0^2}{2}\right) \frac{T^2}{T_1^2 - T^2}$
    \item[\bf Ход работы.] Результаты вычислений представлены в таблице.
\end{enumerate}


\renewcommand{\arraystretch}{1.5}
\setlength{\tabcolsep}{2pt}
\begin{longtable}[h!,leftmargin=-2cm]{ |C{1.5cm}|c|c|c|c|c|c|c|C{1.5cm}|C{1.5cm}|C{1.5cm}| }
    \hline
    {№ опыта} & $t$, с & $\overline t$, с & $\Delta t$, с & $\overline{\Delta t}$, с & $h$, см & $m$, г & $d$, мм & $I, \text{кг}\cdot\text{м}^2$ & $M_1, \text{Н}\cdot\text{м}$ & $M_2, \text{Н}\cdot\text{м}$ \\
    \hline
    \endhead
1 & 5,388 & \multirow{3}{*}{5.198} & 0,190 & \multirow{9}{*}{0,075} & \multirow{9}{*}{45} & \multirow{3}{*}{49,4} & \multirow{9}{*}{300} & \multirow{9}{*}{0,038} & \multirow{3}{*}{0,072} & \multirow{3}{*}{0,000}\\
\cline{1-2} \cline{4-4} 2 & 5,081 & ~ & 0,117 & ~ & ~ & ~ & ~ & ~ & ~ & ~\\
\cline{1-2} \cline{4-4} 3 & 5,125 & ~ & 0,073 & ~ & ~ & ~ & ~ & ~ & ~ & ~\\
\cline{1-4} \cline{7-7} \cline{10-11} 4 & 4,102 & \multirow{3}{*}{4.059} & 0,043 & ~ & ~ & \multirow{3}{*}{99,20} & ~ & ~ & \multirow{3}{*}{0,145} & \multirow{3}{*}{0,001}\\
\cline{1-2} \cline{4-4} 5 & 4,114 & ~ & 0,055 & ~ & ~ & ~ & ~ & ~ & ~ & ~\\
\cline{1-2} \cline{4-4} 6 & 3,961 & ~ & 0,098 & ~ & ~ & ~ & ~ & ~ & ~ & ~\\
\cline{1-4} \cline{7-7} \cline{10-11} 7 & 3,504 & \multirow{3}{*}{3.453} & 0,051 & ~ & ~ & \multirow{3}{*}{149,10} & ~ & ~ & \multirow{3}{*}{0,217} & \multirow{3}{*}{0,002}\\
\cline{1-2} \cline{4-4} 8 & 3,407 & ~ & 0,046 & ~ & ~ & ~ & ~ & ~ & ~ & ~\\
\cline{1-2} \cline{4-4} 9 & 3,448 & ~ & 0,005 & ~ & ~ & ~ & ~ & ~ & ~ & ~\\
    \hline
\end{longtable}

Относительная погрешность составляет .

Максимальная погрешность составляет .

\begin{enumerate}[leftmargin=3.5cm]
    \item[\bf Вывод Данилы.]
    \item[\bf Вывод Вики.] 
\end{enumerate}


\section{Контрольные вопросы}
\section{Сформулируйте основное уравнение вращательного движения твёдрого тела}



\end{document}
