\documentclass[otchet]{SCWorks}
% Тип обучения (одно из значений):
%    bachelor   - бакалавриат (по умолчанию)
%    spec       - специальность
%    master     - магистратура
% Форма обучения (одно из значений):
%    och        - очное (по умолчанию)
%    zaoch      - заочное
% Тип работы (одно из значений):
%    coursework - курсовая работа (по умолчанию)
%    referat    - реферат
%  * otchet     - универсальный отчет
%  * nirjournal - журнал НИР
%  * digital    - итоговая работа для цифровой кафдры
%    diploma    - дипломная работа
%    pract      - отчет о научно-исследовательской работе
%    autoref    - автореферат выпускной работы
%    assignment - задание на выпускную квалификационную работу
%    review     - отзыв руководителя
%    critique   - рецензия на выпускную работу
% Включение шрифта
%    times      - включение шрифта Times New Roman (если установлен)
%                 по умолчанию выключен
\usepackage[T2A]{fontenc}
\usepackage[utf8]{inputenc}
\usepackage{graphicx}
\usepackage[sort,compress]{cite}
\usepackage{amsmath}
\usepackage{amssymb}
\usepackage{amsthm}
\usepackage{fancyvrb}
\usepackage{longtable}
\usepackage{array}
\usepackage[english,russian]{babel}
\usepackage{minted}
\usepackage{tempora}
\usepackage[hidelinks]{hyperref}


\usepackage{multirow}
\usepackage[table]{xcolor}\usepackage{longtable}\usepackage{array}
\usepackage{graphicx}%Вставка картинок правильная

\usepackage{float}%"Плавающие" картинки

\usepackage{wrapfig}%Обтекание фигур (таблиц, картинок и прочего)
\setlength{\arrayrulewidth}{0.5mm}
\setlength{\tabcolsep}{18pt}

\newenvironment{centeritemize*}[1][]
  {\par\centering\begin{itemize*}[itemjoin=\quad,#1]}
  {\end{itemize*}\par}
\newcolumntype{C}[1]{>{\centering\let\newline\\\arraybackslash\hspace{0pt}}m{#1}}


\begin{document}

% Кафедра (в родительном падеже)
\chair{математической кибернетики и компьютерных наук}

% Тема работы
\title{Измерение моментов инерции тел}

% Курс
\course{1}

% Группа
\group{151}

% Факультет (в родительном падеже) (по умолчанию "факультета КНиИТ")
% \department{факультета КНиИТ}

% Специальность/направление код - наименование
% \napravlenie{02.03.02 "--- Фундаментальная информатика и информационные технологии}
% \napravlenie{02.03.01 "--- Математическое обеспечение и администрирование информационных систем}
% \napravlenie{09.03.01 "--- Информатика и вычислительная техника}
\napravlenie{09.03.04 "--- Программная инженерия}
% \napravlenie{10.05.01 "--- Компьютерная безопасность}

% Для студентки. Для работы студента следующая команда не нужна.
\studenttitle{Студентов}

% Фамилия, имя, отчество в родительном падеже
\author{Григорьева Данилы Евгеньевича и Лазаревой Виктории Владимировны}

% Заведующий кафедрой 
\chtitle{доцент, к.\,ф.-м.\,н.}
\chname{С.\,В.\,Миронов}

% Руководитель ДПП ПП для цифровой кафедры (перекрывает заведующего кафедры)
% \chpretitle{
%     заведующий кафедрой математических основ информатики и олимпиадного\\
%     программирования на базе МАОУ <<Ф"=Т лицей №1>>
% }
% \chtitle{г. Саратов, к.\,ф.-м.\,н., доцент}
% \chname{Кондратова\, Ю.\,Н.}

% Научный руководитель (для реферата преподаватель проверяющий работу)
\satitle{доцент, к.\,ф.-м.\,н.} %должность, степень, звание
\saname{О.\,А.\,Черкассова}

% Руководитель практики от организации (руководитель для цифровой кафедры)
\patitle{доцент, к.\,ф.-м.\,н.}
\paname{С.\,В.\,Миронов}

% Руководитель НИР
\nirtitle{доцент, к.\,п.\,н.} % степень, звание
\nirname{В.\,А.\,Векслер}

% Семестр (только для практики, для остальных типов работ не используется)
\term{2}

% Наименование практики (только для практики, для остальных типов работ не
% используется)
\practtype{учебная}

% Продолжительность практики (количество недель) (только для практики, для
% остальных типов работ не используется)
\duration{2}

% Даты начала и окончания практики (только для практики, для остальных типов
% работ не используется)
\practStart{01.07.2022}
\practFinish{13.01.2023}

% Год выполнения отчета
\date{2023}

\maketitle

\secNumbering

\newpage

\begin{enumerate}[leftmargin=3.5cm]
    \item[\bf Наименование работы.] Измерение моментов инерции тел.
    \item[\bf Цель.] Изучить понятия момента инерции материальной  точки и твёрдого тела, ознакомиться с экспериментальными способами измерения моментов инерции тел, проверка теоремы Штейнера-Гюйгенса, оценка точности методов измерения.
    \item[\bf Принадлежности.] Крутильный маятник, секундомер, линейка, набор грузов, исследуемые тела.
    \item[\bf Рабочая формула.] $I = \frac m 2 \left( L^2 + \frac{d_0^2}{2}\right) \frac{T^2}{T_1^2 - T^2}$
    \item[\bf Ход работы.] Результаты вычислений представлены в таблице.
\end{enumerate}

\renewcommand{\arraystretch}{1.5}
\setlength{\tabcolsep}{2pt}
\begin{longtable}[h!,leftmargin=-2cm]{ |C{1.5cm}|c|c|c|c|c|c|c|c|C{2.5cm}|C{2.5cm}| }
    \hline
    {№ опыта} & $t$, с & n & $T$, с & $\overline T$, c & $t_1$, с & $n_1$ & $T_1$, c & $\overline T_1$, с & {$K$, $\text{г}\cdot\text{см}^3$} & {$I_0, \text{г}\cdot\text{см}^3$} \\
    \hline
    \endhead
1 & 35,830 & \multirow{3}{*}{30} & 1,194 & \multirow{3}{*}{1,136} & 72,000 & \multirow{3}{*}{30} & 2,400 & \multirow{3}{*}{2,400} & \multirow{3}{*}{26 134,750} & \multirow{3}{*}{7545,981}\\
\cline{1-2} \cline{4-4} \cline{6-6} \cline{8-8}2 & 31,620 & ~ & 1,054 & ~ & 72,050 & ~ & 2,401 & ~ & ~ & ~\\
\cline{1-2} \cline{4-4} \cline{6-6} \cline{8-8}3 & 34,800 & ~ & 1,160 & ~ & 72,360 & ~ & 2,412 & ~ & ~ & ~\\
    \hline
\end{longtable}

\begin{longtable}[h!,leftmargin=-2cm]{ |C{1.5cm}|c|c|c|c|c|c|c|c|C{2.5cm}|C{2.5cm}| }
    \hline
    {№ опыта} & $t$, с & n & $T$, с & $\overline T$, c & $t_1$, с & $n_1$ & $T_1$, c & $\overline T_1$, с & {$K$, $\text{г}\cdot\text{см}^3$} & {$I_1, \text{г}\cdot\text{см}^3$} \\
    \hline
    \endhead
   1 & 46,230 & \multirow{3}{*}{30} & 1,541 & \multirow{3}{*}{1,550} & 77,540 & \multirow{3}{*}{30} & 2,585 & \multirow{3}{*}{2,585} & \multirow{3}{*}{26 134,750} & \multirow{3}{*}{14 671,208}\\
\cline{1-2} \cline{4-4} \cline{6-6} \cline{8-8}2 & 46,300 & ~ & 1,543 & ~ & 77,680 & ~ & 2,589 & ~ & ~ & ~\\
\cline{1-2} \cline{4-4} \cline{6-6} \cline{8-8}3 & 47,000 & ~ & 1,567 & ~ & 77,44 & ~ & 2,581 & ~ & ~ & ~\\
    \hline
\end{longtable}

$m = 107$ г, $L = 22,000$ см, $d_0 = 3,000$ см

Инерция тела относительно его геометрической оси:
\begin{equation*}
    I_\tau = I_1 - I_0 = 7125,228 ~ \text{г} \cdot \text{см}^3 
\end{equation*}

\begin{enumerate}[leftmargin=3.5cm]
    \item[\bf $\substack{\textbf{\normalsize Инструментальная}\\\textbf{\normalsize погрешность.}}$]
    Абсолютная инструментальная погрешность для периода $T$:
    \begin{equation*}
        \pm \frac{0,010}{30 \cdot 2} \approx \pm 0,005 \, \text{с}
    \end{equation*}
    Абсолютная инструментальная погрешность для расстояния $L$ и  диаметра цилиндриков $d_0$:
    \begin{equation*}
        \Delta L = \Delta d_0 = \pm \frac{0,100 \, \text{см}}{2} \approx \pm 0,050 \, \text{см}
    \end{equation*}
    \item[\bf $\substack{\textbf{\normalsize Погрешность}\\\textbf{\normalsize измерения.}}$]
    Относительная погрешность для расстояния $L$:
    \begin{equation*}
        \delta_L = \pm\frac{\Delta L}{L}=\pm\frac{0,050\,\text{см}}{22,000\,\text{см}}\approx\pm0,002\,\text{см}
    \end{equation*}
    Относительная погрешность для диаметра цилиндриков $d_0$:
    \begin{equation*}
        \delta_{d_0} = \pm \frac{\Delta d_0}{3,000\,\text{см}} \approx 0,17\,\text{см}
    \end{equation*}
    Погрешность измерения для периода:
    \begin{equation*}
        \Delta T = \pm \frac{\overline \Delta}{2} = \pm \frac{0,011\,\text{с}}{2} \approx \pm 0,0055 \, \text{с}
    \end{equation*}
    \begin{equation*}
        \delta_T = \pm \frac{\Delta T}{\overline T} \cdot 100\% = \pm \frac{0,0055}{1,550} \cdot 100\% \approx \pm 0,0036 \cdot 100\% = \pm 0,36\%
    \end{equation*}
    \item[\bf Вывод Данилы.]
    В результате поставленных опытов с помощью крутильного маятника был успешно измерен момент инерции материальной точки твёрдого тела. Это позволяет сделать вывод, что теорема Штейнера-Гюйгенса выполняется.
    
    Погрешность результата измерения незначительно превышает инструментальную погрешность из-за некоторых особенностей проведения опыта. Несмотря на этот факт, полученные итоги опыта совпадают с ожидаемыми.
    
    В будущем, чтобы получить более точные результаты, требуется выполнять измерения времени с помощью автоматизированных систем, что поможет избежать временных затрат на скорость человеческой реакции. 
    \item[\bf Вывод Виктории.]
    По проведении экспериментов был получен навык измерения момента инерции материальной точки находящегося в твердом агрегатном состоянии тела посредством крутильного маятника, что позволяет считать правдой утверждение о выполнении теоремы Штейнера-Гюйгенса.
    
    Человеческий фактор, а именно реакция на процессы, за счёт времени, затрачиваемого на конкретные действия, будет влиять на погрешность, в следствие можно прийти к выводу, что необходимо автоматизировать этот процесс посредством приборов.

\end{enumerate}

\end{document}
